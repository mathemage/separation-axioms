\chapter*{Conclusion}
\addcontentsline{toc}{chapter}{Conclusion}

Point-free topology has increasingly important role in modern computer science
and other branches of~mathematics.
Therefore, it is crucial to know which concepts and notions of~point-set and
point-free topologies can be unified and how to do it.

In this treatise we observed that the~relationships of~$T$-axioms in~classic
and in~pointless topologies are alike:
while for classic axioms it holds that
\[
  (T_4) \& (T_1)
  \; \Longrightarrow \; (T_{3\frac{1}{2}}) \& (T_1)
  \; \Longrightarrow \; (T_3) \& (T_1)
  \; \Longrightarrow \; (T_2)
  \; \Longrightarrow \; (T_1)
  \; \Longrightarrow \; (T_0),
\]
for their point-free counterparts we have
\begin{align*}
  \text{(Norm)} \& \text{(Sfit)}
  \; &\Rightarrow \; \text{(CReg)}
  \; \Rightarrow \; \text{(Reg)}
  \, \Rightarrow \; \text{(I-Haus)} \& \text{(Sfit)}
  \; \Rightarrow \\
  \; &\Rightarrow \; \text{(DS-Haus)} \& \text{(Sfit)}
  \; \Rightarrow \; \text{all axioms from chapter IV}.
\end{align*}

It may seem intriguing to also study pointless variants of~other separations
(not only the family of~$T$-axioms).
Nonetheless, such a~task is beyond the scope of~this bachelor thesis and it is
a~potential topic for other works.
