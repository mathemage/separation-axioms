\chapter{Subfitness}

\section{The $T_1$ axiom}

\begin{df}[$T_1$]
  A topological space $(X, \tau)$ is said to be a {\sl $T_1$-space\/} (or,
  equivalently, to satisfy the {\sl $T_1$ axiom)\/} if the following assertion
  holds:

  \begin{center}
    for any $x \ne y$ from $X$ there exists an open set $U \in \tau$ such that $x
    \in U \not\owns y$.
  \end{center}
\end{df}

The next lemma will be of great usefulness later in this chapter.

\begin{lem}
  A topological space $X$ is $T_1$ iff any of its finite subsets is closed.
\end{lem}

\begin{proof}
  Choose an arbitrary $S \Subset X$.

  $\Rightarrow$:

  $\Leftarrow$:
\end{proof}

It is noteworthy to mention the possibility of formulating $T_1$-counterparts
in the context of pointless topology.
However, such efforts tend to enforce {\sl spatiality\/} --- a property not
always desirable.
Instead, we may introduce a weaker yet still beneficial condition: the subfit
axiom.

\section{The subfit axiom}

\begin{df}[Sfit]
  A locale $L$ is said to be {\sl subfit\/} whenever

  \[
    a \not\le b \qquad \Rightarrow \qquad \exists c: \quad a \vee c = 1 \ne b
    \vee c.
  \]
\end{df}

Similarly, subfitness of a topological space $(X, \Omega(X))$ amounts to
requiring that the locale $\Omega(X)$ is subfit.

Such a condition is, as shown below, truly weaker than the $T_1$ axiom.

\begin{prop}
  Every $T_1$-space is subfit.
\end{prop}

\begin{proof}
  For each $T_1$-space $X$ the locale $\Omega(X)$ is subfit.
  Indeed: for open $A \not\subseteq B$ there exists an $x \in A \setminus B$.
\end{proof}
