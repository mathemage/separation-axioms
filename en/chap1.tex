\chapter{Subfitness}

\section{The $T_1$ axiom}

\begin{framed}
  \begin{df}[$T_1$]
    A topological space $(X, \tau)$ is said to be a \emph{$T_1$-space\/} (or,
    equivalently, to satisfy the \emph{$T_1$ axiom\/}) if the following assertion
    holds:
    \begin{center} \it
      for any $x \ne y$ from $X$ there exists an open set $U\in \tau$ such that
      $x\in U \not\owns y$.
    \end{center}
  \end{df}
\end{framed}

The next lemma will be of great usefulness later in this chapter.

\begin{lem} \label{T1Char}
  A topological space $X$ is $T_1$ iff $\overline{\left\{x\right\}} =
  \left\{x\right\}$ for any $x\in X$.
\end{lem}

\begin{proof}
  $\Rightarrow$: If $y\not\in \left\{x\right\}$, we obtain, guaranteed by the
  $T_1$ axiom, an open $U$ such that $y\in U\not\owns x$.
  In other words, an open set containing $y$ albeit not intersecting
  the $\left\{x\right\}$.
  Thus, $y\not\in \overline{\left\{x\right\}}$, and consequently,
  $\left\{x\right\} = \overline{\left\{x\right\}}$.

  $\Leftarrow$: With open $U:= X\setminus \left\{y\right\}$ we can easily
  check the $T_1$.
\end{proof}

\begin{cor}
  A space is $T_1$ iff every of its finite subsets is closed.
\end{cor}

\begin{proof}
  The aforementioned subset is a finite union of its elements---each of them
  closed.
  Hence, also closed itself.

  The other implication is trivial.
\end{proof}

It is noteworthy to mention the possibility of formulating $T_1$-counterparts
in~the~context of pointless topology.
Nonetheless, such efforts tend to enforce {\sl spatiality\/}: a property not
all the time desirable.
Instead, we may introduce a weaker yet still beneficial condition: {\sl the
subfitness\/}.

\section{The subfit axiom}

\begin{framed}
  \begin{df}[Sfit]
    A locale $L$ is said to be \emph{subfit\/} whenever
    \[
      a \not\le b \qquad \Rightarrow \qquad \exists c: \quad a \vee c = 1 \ne b
      \vee c.
    \]
  \end{df}
\end{framed}

Similarly, subfitness of a topological space $(X, \Omega(X))$ amounts to
requiring that the locale $\Omega(X)$ is subfit.

Such a condition is, as shown below, truly weaker than the $T_1$ axiom.

\begin{prop} \label{T1->Sfit}
  Every $T_1$-space is subfit.
\end{prop}

\begin{proof}
  For each $T_1$-space $X$ the corresponding locale $\Omega(X)$ is subfit.
  Indeed: for open $A\not\subseteq B$ there exists an $x\in A \setminus B$.
  By Lemma~\ref{T1Char}\thinspace, we have an open $X\setminus
  \left\{x\right\}$.
  Since $x\in A$ and $x\not\in B$, we conclude with $A\cup (X\setminus
  \left\{x\right\}) = X \ne B \cup (X\setminus \left\{x\right\})$.
\end{proof}

Note that the reverse implication does not hold.
In search of a convenient counterexample we need to further discuss the form
which the subfitness takes in~topological spaces.

\begin{prop}
  A space $X$ is subfit iff any non-empty difference $A\setminus B$ of~closed
  sets contains a non-void closed set.
\end{prop}

\begin{proof}
  Let us write $U$ for $X\setminus B$ and $V$ for $X\setminus A$.
  Obviously,
  \[
    A\subseteq B \; \Leftrightarrow \; U\subseteq V,
  \]
  which can be rewritten as a useful equivalence
  \[
    A\not\subseteq B \; \Leftrightarrow \; U\not\subseteq V.
  \]

  $\Rightarrow$: If $A\setminus B \ne \none$, then $A\not\subseteq B$.
  Whence, $U\not\subseteq V$ and, using subfitness, we gain an~open $W \ne X$
  with the property $U \cup W = X \ne V \cup W$. 
  Without loss of generality, we may assume $V\subseteq W$: for the open set $V
  \cup W$ meets the same requirements, as well.

  With $C := X \setminus W$ one has the desired non-void closed set
  in~$A\setminus B$.
  Firstly, $V\subseteq W$ implies $C\subseteq A$.
  Secondly, $U\not\subseteq W$ (since otherwise $W = U \cup W = X$ and,
  in~conclusion, $V \cup W = X$).
  Ergo, $C\not\subseteq B$.

  $\Leftarrow$: Considering $U$ and $V$ to be open, the complements $A$ and $B$
  are closed.
  Provided that $U\not\subseteq V$, we have $A\not\subseteq B$.
  Therefore, $A \setminus B$ as a~non-empty difference contains a~closed
  set~$C \ne \none$.

  Setting $W := X \setminus C$, we gather that $W \cup U = X$ (because $U =
  \minus B \supseteq C$) and finally $W \cup V \ne X$ (as $C\subseteq A =
  \minus V$ leads to non-empty $C\not\subseteq W \cup V$).
\end{proof}

\begin{rem}
  The previous characterization of (Sfit) gives rise to~another proof of
  Proposition~\ref{T1->Sfit}\thinspace:
  any singleton in a non-empty difference is, by~\ref{T1Char}\thinspace,
  already closed.
\end{rem}

Now we will demonstrate the promised example of a subfit non-$T_1$-space.

\begin{exmpl}
  Let us have $X = \N \cup \left\{\infty\right\}$ and $\theta = \left\{
  F\subseteq X \st X\setminus F \Subset \N \right\} \cup \left\{ \none
  \right\}$.
  That is, $\mathcal{X} := (X, \theta)$ is the topological space where closed
  sets consist of~finite sets of~natural numbers and of~the whole space $X$.

  It is not $T_1$ for the one-point set $\left\{ \infty \right\}$ is not
  closed (see Lemma~\ref{T1Char}).

  On the other hand, there always is a closed $C \subseteq A\setminus B$ for
  arbitrary closed $A\not\subseteq B$:
  if both are finite, so is their difference.
  Otherwise one of them equals the whole $X$ and---since $A\setminus X$ would
  be empty---it has to be the $A$.
  Then, apparently, a singleton from $\N\setminus B \subseteq X\setminus B$ is
  again closed, and thusly, the subfitness of~$\mathcal{X}$ becomes evident.
\end{exmpl}
