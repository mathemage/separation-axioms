\chapter{Subfitness}

\section{The $T_1$ axiom}

\begin{df}[$T_1$]
  A topological space $(X, \tau)$ is said to be a {\sl $T_1$-space\/} (or,
  equivalently, to satisfy the {\sl $T_1$ axiom)\/} if the following assertion
  holds:
  \begin{center}
    for any $x \ne y$ from $X$ there exists an open set $U\in \tau$ such that
    $x\in U \not\owns y$.
  \end{center}
\end{df}

The next lemma will be of great usefulness later in this chapter.

\begin{lem} \label{T1Char}
  A topological space $X$ is $T_1$ iff any of its finite subsets is closed.
\end{lem}

\begin{proof}
  Let $F \Subset X$.

  $\Rightarrow$: Trivially, $F = \emptyset$ is closed.
  
  If $z\not\in F$, we obtain, granted by the $T_1$ axiom, $U_x$
  for each $x\in F$ such that $z\in U_x\not\owns x$.
  We denote $U := \bigcap_{x\in F}U_x$.
  Then $U$ is, as a finite intersection of~open sets, open itself, and
  moreover, $z\in U$ and $U \cap F = \emptyset$.
  Thus, $z\not\in \overline{F}$, and consequently, $F = \overline{F}$ is
  closed.

  $\Leftarrow$: As the finite $\{y\}$ is closed, we have $x\in X\setminus
  \{y\} \not\owns y$ for open $X\setminus \{y\}$.
\end{proof}

It is noteworthy to mention the possibility of formulating $T_1$-counterparts
in the context of pointless topology.
However, such efforts tend to enforce {\sl spatiality\/}: a property not always
desirable.
Instead, we may introduce a weaker yet still beneficial condition: {\sl the
subfit axiom\/}.

\section{The subfit axiom}

\begin{df}[Sfit]
  A locale $L$ is said to be {\sl subfit\/} whenever
  \[
    a \not\le b \qquad \Rightarrow \qquad \exists c: \quad a \vee c = 1 \ne b
    \vee c.
  \]
\end{df}

Similarly, subfitness of a topological space $(X, \Omega(X))$ amounts to
requiring that the locale $\Omega(X)$ is subfit.

Such a condition is, as shown below, truly weaker than the $T_1$ axiom.

\begin{prop}
  Every $T_1$-space is subfit.
\end{prop}

\begin{proof}
  For each $T_1$-space $X$ the corresponding locale $\Omega(X)$ is subfit.
  Indeed: for open $A\not\subseteq B$ there exists an $x\in A \setminus B$.
  By Lemma~\ref{T1Char}\thinspace, we have an open $X\setminus \{x\}$.
  Since $x\in A$ and $x\not\in B$, we conclude with $A\cup (X\setminus \{x\}) =
  X \ne B \cup (X\setminus \{x\})$.
\end{proof}
