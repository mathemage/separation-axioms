\chapter{Subfitness}

\section{The $T_1$ axiom}

\begin{df}[$T_1$]
  A topological space $(X, \tau)$ is said to be a {\sl $T_1$-space\/} (or,
  equivalently, to satisfy the {\sl $T_1$ axiom)\/} if the following assertion
  holds:
  \begin{center}
    for any $x \ne y$ from $X$ there exists an open set $U\in \tau$ such that
    $x\in U \not\owns y$.
  \end{center}
\end{df}

The next lemma will be of great usefulness later in this chapter.

\begin{lem} \label{T1Char}
  A topological space $X$ is $T_1$ iff $\overline{\{x\}} = \{x\}$ for any $x\in
  X$.
\end{lem}

\begin{proof}
  $\Rightarrow$: If $y\not\in \{x\}$, we obtain, granted by the $T_1$ axiom,
  an open $U$ such that $y\in U\not\owns x$.
  In other words, an open set containing $y$ though not intersecting $\{x\}$.
  Thus, $y\not\in \overline{\{x\}}$, and consequently, $\{x\} =
  \overline{\{x\}}$.

  $\Leftarrow$: With open $U:= X\setminus \{y\}$ we can easily verify the
  $T_1$.
\end{proof}

\begin{cor}
  A space is $T_1$ iff every of its finite subsets is closed.
\end{cor}

\begin{proof}
  The aforementioned subset is a finite union of its elements -- each of them closed.
  Hence, also closed itself.

  The other implication is trivial.
\end{proof}

It is noteworthy to mention the possibility of formulating $T_1$-counterparts
in the context of pointless topology.
However, such efforts tend to enforce {\sl spatiality\/}: a property not always
desirable.
Instead, we may introduce a weaker yet still beneficial condition: {\sl the
subfit axiom\/}.

\section{The subfit axiom}

\begin{df}[Sfit]
  A locale $L$ is said to be {\sl subfit\/} whenever
  \[
    a \not\le b \qquad \Rightarrow \qquad \exists c: \quad a \vee c = 1 \ne b
    \vee c.
  \]
\end{df}

Similarly, subfitness of a topological space $(X, \Omega(X))$ amounts to
requiring that the locale $\Omega(X)$ is subfit.

Such a condition is, as shown below, truly weaker than the $T_1$ axiom.

\begin{prop}
  Every $T_1$-space is subfit.
\end{prop}

\begin{proof}
  For each $T_1$-space $X$ the corresponding locale $\Omega(X)$ is subfit.
  Indeed: for open $A\not\subseteq B$ there exists an $x\in A \setminus B$.
  By Lemma~\ref{T1Char}\thinspace, we have an open $X\setminus \{x\}$.
  Since $x\in A$ and $x\not\in B$, we conclude with $A\cup (X\setminus \{x\}) =
  X \ne B \cup (X\setminus \{x\})$.
\end{proof}

Note that the reverse implication does not hold.
In search of a convenient counterexample we need to further discuss the form
which the subfitness takes in~topological spaces.

\begin{prop}
  A space $X$ is subfit iff any non-empty difference $A\setminus B$ of~closed
  sets contains a non-empty closed set.
\end{prop}

\begin{proof}
  Let us write $U$ for $X\setminus B$ and $V$ for $X\setminus A$.
  Obviously,
  \[
    A\subseteq B \; \Leftrightarrow \; U\subseteq V,
  \]
  which can be rewritten as a useful statement
  \[
    A\not\subseteq B \; \Leftrightarrow \; U\not\subseteq V.
  \]

  $\Rightarrow$: If $A\setminus B \ne \emptyset$, then $A\not\subseteq B$.
  Whence, $U\not\subseteq V$ and, using subfitness, we gain an~open $W$ with
  the property $U \cup W = X \ne V \cup W$. 
  Without loss of generality, we may assume $V\subseteq W$: for the open set $V
  \cup W$ meets the same requirements.

  With $C := X \setminus W$ one has the desired closed set in~$A\setminus B$.
  Firstly, $V\subseteq W$ implies $C\subseteq A$.
  Secondly, $U\not\subseteq W$ (since otherwise $W = U \cup W = X$ and,
  in~conclusion, $V \cup W = X$).
  Ergo, $C\not\subseteq B$.

  $\Leftarrow$: Considering $U$ and $V$ to be open, the complements $A$ and $B$
  are closed.
  Provided that $U\not\subseteq V$, we have $A\not\subseteq B$.
  Therefore, $A \setminus B$ as a~non-empty difference contains a~closed
  set~$C$.

  Setting $W := X \setminus C$, we gather that $W \cup U = X$ (because $U =
  \minus B \supseteq C$) and finally $W \cup V \ne X$ (as $C\subseteq A =
  \minus V$ leads to $C\not\subseteq W \cup V$).
\end{proof}
