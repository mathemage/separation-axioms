\chapter{Subfitness}

\section{The $T_1$ axiom}

\begin{framed}
  \begin{df}[$T_1$]
    A topological space $(X, \tau)$ is said to be a \emph{$T_1$-space\/} (or to
    satisfy the \emph{$T_1$ axiom\/}) if the following assertion holds:
    \begin{center} \it
      for any $x \ne y$ from $X$ there exists an open set $U\in \tau$ such that
      $x\in U \not\owns y$.
    \end{center}
  \end{df}
\end{framed}

The following is a standard and useful characterization of this property.

\begin{fact-num} \label{T1Char}
  A topological space $X$ is $T_1$ iff $\overline{\left\{x\right\}} =
  \left\{x\right\}$ for any $x\in X$.
\end{fact-num}

(Of course:
if $y\not\in \left\{x\right\}$, we obtain an open $U$ such that $y\in
U\not\owns x$, and thus, $y\not\in \overline{\left\{x\right\}}$.
Conversely, with open $U:= X\setminus \left\{y\right\}$ we can easily check the
$T_1$.)

\begin{cor}
  A space is $T_1$ iff every of its finite subsets is closed.
\end{cor}

There are exact counterparts of this axiom in the point-free context (see,
e.~g.,~\cite{ds72}).
They have not found much use so far, though.
Instead, one introduces a weaker and very useful condition: {\sl the
subfitness\/}.

\section{The subfit axiom}

\begin{framed}
  \begin{df}[Sfit]
    A locale $L$ is said to be \emph{subfit\/} whenever
    \[
      a \not\le b \qquad \Rightarrow \qquad \exists c: \quad a \vee c = 1 \ne b
      \vee c.
    \]
  \end{df}
\end{framed}

\begin{thm} \label{T1->Sfit}
  Every $T_1$-space is subfit.
\end{thm}

\begin{proof}
  For each $T_1$-space $X$ the corresponding locale $\Omega(X)$ is subfit.
  Indeed: for open $A\not\subseteq B$ there exists an $x\in A \setminus B$.
  By~\ref{T1Char}\thinspace, we have an open $X\setminus \left\{x\right\}$.
  Since $x\in A$ and $x\not\in B$, we conclude with $A\cup (X\setminus
  \left\{x\right\}) = X \ne B \cup (X\setminus \left\{x\right\})$.
\end{proof}

Another characterization of the subfitness:

\begin{prop}[Isbell, Simmons] \label{Sfit-char}
  For a topological space $X$ the locale $\Omega(X)$ is subfit iff
  \[
    \forall U\in\Omega(X) \forall x\in U \exists y\in \overline{\{x\}}: \quad
    \overline{\{y\}} \subseteq U
  \]
\end{prop}
\begin{proof}
  TBD
\end{proof}

There is an example of a subfit non-$T_1$-space.

\begin{exmpl}
  Let us have $X = \N \cup \left\{\infty\right\}$ and $\theta = \left\{
  F\subseteq X \st X\setminus F \Subset \N \right\} \cup \left\{ \none
  \right\}$.
  That is, $\mathcal{X} := (X, \theta)$ is the topological space where closed
  sets consist of~finite sets of~natural numbers and of~the whole space $X$.

  It is not $T_1$ for the one-point set $\left\{ \infty \right\}$ is not
  closed (viz \ref{T1Char}).

  On the other hand, there always is a closed $C \subseteq A\setminus B$ for
  arbitrary closed $A\not\subseteq B$:
  if both are finite, so is their difference.
  Otherwise one of them equals the whole $X$ and---since $A\setminus X$ would
  be empty---it has to be the $A$.
  Then, apparently, a singleton from $\N\setminus B \subseteq X\setminus B$ is
  again closed, and thusly, the subfitness of~$\mathcal{X}$ becomes evident.
\end{exmpl}
