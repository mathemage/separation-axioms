\chapter*{Introduction}
\addcontentsline{toc}{chapter}{Introduction}

In classical topology, spaces are often specified by \emph{separation axioms\/}
of~various strength.
The stronger axiom we have, the more ``geometrical'' a~space appears to be.

The best-known of~them are the~$T_i$-axioms ($i = 0, 1, 2, 3, 3\frac{1}{2},
4$).
They concern the separation of~points, points from closed sets, and closed sets
from each other.
The role of~points seems to be fundamental, and therefore, it may seem hard to
find natural counterparts of the $T$-axioms in the point-free context.
Nevertheless, most of them do have pointless equivalents.
The aim of this thesis is to provide their summary and to show their mutual
relationships.

\section*{Structure of thesis}

The first chapter contains the necessary concepts of~order and category
theories, followed by the essentials of~point-set and point-free topology.

In the subsequent chapter we discuss the \emph{subfitness\/}---a~weakened form
of~the $T_1$-axiom (the $T_1$-axiom itself does not seem to have a~very natural
counterpart).
It should be noted that the subfitness is crucial for some relationships
between other axioms.
Besides, it makes good sense in~classical topology as well, and moreover, plays
a~role in~logic.

The \emph{Hausdorff\/} axiom ($T_2$) does not have a~unique direct counterpart.
Unlike in~the other cases, one has several non-equivalent alternatives.
The two most standard ones are discussed in chapter III and in chapter IV we
present a~survey of~several others.

Chapters V and VI are devoted to the \emph{regularity\/} and \emph{complete
regularity\/}.
We show how to exclude points from their definitions via ``rather-below''
relation $\rb$ and ``completely-below'' relation $\cb$, and thus, how to create
their point-free analogies.

The final chapter concerns \emph{normality\/}, the translation of~which is
perhaps the easiest one.
We discuss the relation with the other axioms.
