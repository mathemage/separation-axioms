\chapter*{Introduction}
\addcontentsline{toc}{chapter}{Introduction}

In classic topology, spaces are often specified by \emph{separation axioms\/}
of~various strength.
The stronger axiom we have, the more ``geometrical'' a~space appears to be.

One of~the best-known series is the family of~$T$-axioms.
They concern the separation of~points, points from closed sets, and closed sets
from other closed sets.
In~this context the role of~points is fundamental:
it may seem immensely difficult to~imitate $T$-axioms in~point-free topology.
Nevertheless, most of them do have pointless counterparts and this thesis deals
with them.

\section*{Structure of thesis}

The first chapter provides basics of~order and category theories later followed
by~essentials of~point-set and point-free topologies.

The subsequent chapter presents \emph{subfitness\/}---a~weakened form of~the
$T_1$-axiom, which itself is not interesting in the point-free context, whereas
the subfitness is crucial for relations between other axioms.

Axioms of~\emph{Hausdorff type\/} have a~distinctive role:
unlike the others, we are not able to directly translate them from the
$T_2$-axiom.
For various reasons point-free versions cannot correspond to the point-set
Hausdorff property.

Chapters V and VI regard \emph{regularity\/} and \emph{complete regularity\/}.
We show how to exclude points from their definitions via ``rather-below''
relation $\rb$, and thus, how to create their point-free analogies.

The final chapter regards \emph{normality\/}, which we can very naturally
translate into the point-free language.
