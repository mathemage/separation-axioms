\chapter{Regularly behaving regularity}

We are going to examine regularity in the pointless context;
how it is stronger than both strongly Hausdorff property and subfitness, and
thus, how it implies all the aforementioned axioms.

\section{The $T_3$ axiom}

\begin{framed}
  \begin{df}[$T_3$]
    We will consider a topological space $X$ to be \emph{regular\/} (or
    satisfying the~\emph{$T_3$-axiom\/}) when
    \begin{center} \it
      for any $x\in X$ and any closed $A \not\owns x$ there are open $V_1\owns
      x$ and $V_2\supseteq A$ disjoint from~one another.
    \end{center}
  \end{df}
\end{framed}

Even though the definition mentions points, it may be elegantly reduced so as
not to do so.

\begin{prop}
  A space $X$ is regular iff each $U\in \Omega(X)$ can be depicted as
  \[
    U = \bigcup \{V\in \Omega(X) \st \overline{V} \subseteq U\}.
  \]
\end{prop}
\begin{proof}
  The inclusion~$\supseteq$ constantly holds (as $V\subseteq
  \overline{V}\subseteq U$).

  $\Rightarrow$:
  Choose any $x\in U$.
  Since $A := X\setminus U \not\owns x$ is closed, the existence
  of~$V_1$~and~$V_2$ from $(T_3)$ is granted.
  The disjointedness $V_1 \cap V_2 = \none$ leads to $V_1\subseteq X\setminus
  V_2$, which is, incidentally, a closed set.
  Therefore, $\overline{V_1}\subseteq X\setminus V_2$.

  Furthermore, the relation $A\subseteq V_2$ is undoubtedly equivalent to the
  relation $X\setminus V_2\subseteq X \setminus A = U$.
  Hence, for $V_x := V_1$ one has $\overline{V_x}\subseteq X\setminus V_2\subseteq
  U$.
  Such a~system $\{V_x \st x\in U\}$ constitutes a~subset cover of $U$, and
  consequently, finishes the proof of~the~inclusion $\subseteq$.

  $\Leftarrow$:
  With $U := X\setminus A$ take $V_x$ from above and set $V_1 := V_x$ and $V_2
  := X\setminus \overline{V_x}$.
\end{proof}
