\chapter{Regularity}

We are going to examine regularity in the pointless context;
how it is stronger than both strongly Hausdorff property and subfitness, and
thus, how it implies all the aforestated axioms.

\section{The $T_3$ axiom}

\begin{framed}
  \begin{df}[$T_3$]
    We will consider a topological space $X$ to be \emph{regular\/} (or
    satisfying the~\emph{$T_3$-axiom\/}) when
    \begin{center} \it
      for any $x\in X$ and any closed $A \not\owns x$ there are open $V_1\owns
      x$ and $V_2\supseteq A$ disjoint from~one another.
    \end{center}
  \end{df}
\end{framed}

Even though the definition mentions points, it may be elegantly reduced so as
not to do so.

\begin{prop} \label{reg-char}
  A space $X$ is regular iff each $U\in \Omega(X)$ can be depicted as
  \[
    U = \bigcup \{V\in \Omega(X) \st \overline{V} \subseteq U\}.
  \]
\end{prop}
\begin{proof}
  The inclusion~$\supseteq$ constantly holds (as $V\subseteq
  \overline{V}\subseteq U$).

  $\Rightarrow$:
  Choose any $x\in U$.
  Since $A := X\setminus U \not\owns x$ is closed, the existence
  of~$V_1$~and~$V_2$ from $(T_3)$ is stipulated.
  The disjointedness $V_1 \cap V_2 = \none$ leads to $V_1\subseteq X\setminus
  V_2$, which is, incidentally, a closed set.
  Therefore, $\overline{V_1}\subseteq X\setminus V_2$.

  Furthermore, the relation $A\subseteq V_2$ is undoubtedly equivalent to the
  relation $X\setminus V_2\subseteq X \setminus A = U$.
  Hence, for $V_x := V_1$ one has $\overline{V_x}\subseteq X\setminus V_2\subseteq
  U$.
  Such a~system $\{V_x \st x\in U\}$ constitutes a~subset cover of $U$, and
  consequently, finishes the proof of~the~inclusion $\subseteq$.

  $\Leftarrow$:
  With $U := X\setminus A$ take $V_x$ from above and set $V_1 := V_x$ and $V_2
  := X\setminus \overline{V_x}$.
\end{proof}

We may even ponder the possibility of eliminating the~closure in
the~description:

\begin{lem}
  $\overline{V}\subseteq U \quad \equiv \quad \exists W\in \Omega(X)\colon W \cap V =
  \none \;\; \& \;\; W \cup U = X$.
\end{lem}
\begin{proof}
  $\Rightarrow$:
  Take $W := X\setminus \overline{V}$.

  $\Leftarrow$:
  Recall closure's definition.
  Suppose $z\in \overline{V}\setminus U$.
  Since $W \cup U = X$, the $z$ must lie in the $W$;
  howbeit, the $W$ does not intersect the $V$---forcing $z\not\in
  \overline{V}$.
\end{proof}

Without loss of generality, the $W$ might be replaced by~the
pseudocomplement~$V^*$;
explicitly, the $X\setminus \overline{V}$ in $\Omega(X)$.%
\footnote{In frames, pseudocomplements are always present: $a^* := a
\rightarrow 0$ exist owing to frame's being Heyting algebra.}

\section{Regular locales}

\begin{framed}
  \begin{nota}[$\rb$]
    \[
      V \rb U \quad \equiv \quad V^* \vee U = 1,
    \]
    which is referred to by pronouncing \emph{``V is rather below U''\/}.
  \end{nota}
\end{framed}

The next auxiliary lemma will be advantageous property for the rest of this
chapter.
\begin{lem} \label{rb->leq}
  $a \rb b \Rightarrow a \leq b$.
\end{lem}
\begin{proof}
  With help of distributivity,
  \[
    a = a \wedge 1 = a \wedge (a^* \vee b) = (a \wedge a^*) \vee (a \wedge b) =
    0 \vee (a \wedge b) = a \wedge b;
  \]
  which is required to proved.
\end{proof}

Possessing $\rb$ notation, we are able to mimic characterization
in~\ref{reg-char} by
\begin{framed}
  \begin{df}[Reg]
    A locale is called \emph{regular\/} if and only if
    \[
      a = \bigvee \{x \st x \rb a\}
    \]
    for all its elements $a$.
  \end{df}
\end{framed}

Moreover, the proposition leads us to
\begin{cor}
  A topological space $X$ is regular iff its $\Omega(X)$ is regular.
\end{cor}

Now we are going to study the stated relationships between (Reg) and other
axioms.

\begin{fact}
  It holds that
  \[
    \bigoplus_{i\in J} a_i = \bigwedge_{i\in J} \iota_i(a_i)
  \]
  where $(\iota_i\colon L \to \bigoplus_{i\in J} L_i)_{i\in J}$ form a
  coproduct of~$(L_i)_{i\in J}$ in~{\bf Frm}.
\end{fact}
(In order to prove the statement, gentle reader would need deeper background
in~frame coproducts:
in particular, the structure of the insertions $(\iota_i)_{i\in J}$.
That is, regrettably, beyond the limits of this material.
Nonetheless, the proof itself may be found as Proposition~IV.5.2.(2) of the
monography~\cite{picado-pultr12}.)

\begin{lem} \label{oplus-vee-distrib}
  ~
  \begin{enumerate}
  \item $\bigvee_{i\in J} \left(a_i \oplus b\right) = \left(\bigvee_{i\in J}
    a_i \right) \oplus b$.
  \item $\bigvee_{i\in J} \left(a \oplus b_i\right) = a \oplus
    \left(\bigvee_{i\in J} b_i \right)$.
  \item $\left(\bigvee_{i\in J} a_i\right) \oplus \left(\bigvee_{j\in J}
    b_j\right) = \bigvee_{i, j\in J} \left(a_i \oplus b_j\right)$.
  \end{enumerate}
\end{lem}
\begin{proof}
  (i):
  By the aforesaid fact,
  \begin{align*}
    \left(\bigvee_{i\in J} a_i \right) \oplus b
    &= \iota_1\left(\bigvee_{i\in J} a_i \right) \wedge \iota_2(b) \\
    &= \left(\bigvee_{i\in J} \iota_1(a_i) \right) \wedge \iota_2(b) \\
    &= \bigvee_{i\in J} \left(\iota_1(a_i) \wedge \iota_2(b) \right)
    = \bigvee_{i\in J} \left(a_i \oplus b\right)
  \end{align*}
  as frame homomorphisms $\iota_i$ preserve suprema (for the second equality)
  and as the binary meet commutes with joins in frames (for the third
  equality).

  (ii):
  Analogously by symmetry.

  (iii):
  By consecutive application of (i) and (ii).
\end{proof}

\begin{lem}
  For a general locale $L$ and any of its saturated $U\in \left\uparrow
  d_L\right.$
  \[
    (a \wedge b, a \wedge b) \in U \qquad \Longrightarrow \qquad \forall x \rb
    a, \, y \rb b: \; (x, y) \in U
  \]
\end{lem}
\begin{proof}
  Beginning with evident $(x, y)\in x \oplus y$, we modify
  \begin{align*}
    x \oplus y &= (x \wedge (y^* \vee b)) \oplus (y \wedge (x^* \vee a)) \\
               &= (\; (x \wedge y^*) \; \vee \; (x \wedge b) \; ) \oplus (\; (y
    \wedge x^*) \; \vee \; (y \wedge a) \; )
  \end{align*}
  (by using ``rather-belowness'' and distributivity, respectively).

  Proceeding with (iii) of Lemma~\ref{oplus-vee-distrib}\thinspace,
  \begin{align*}
     \ldots = \; &(\; (x \wedge y^*) \oplus (y \wedge x^*) \; ) \vee 
            (\; (x \wedge y^*) \oplus (y \wedge a) \; ) \; \vee \\
            \vee \; &(\; (x \wedge b) \oplus (y \wedge x^*) \; ) \vee
            (\; (x \wedge b) \oplus (a \wedge y) \; ) \\
     \leq \; &(y^*\oplus y) \vee (y^*\oplus y) \vee (x\oplus x^*) \vee
                  (\; (a \wedge b)\oplus(a \wedge b) \;)
  \end{align*}
  where the last upper-bound is retrieved from~\ref{rb->leq}\thinspace.

  Besides, the ultimate expression is a subset of $U$:
  as $(x^*\oplus x), (y^*\oplus y)\subseteq d_L$ and $(a \wedge b)\oplus(a
  \wedge b)\subseteq U$ from the premise---all the way leading to $(x, y)\in
  U$.
\end{proof}

The proof of the oncoming theorem will depend upon the reverse implication
of~\ref{meets-in-satur} on page~\pageref{meets-in-satur}\thinspace;
for conciseness of this thesis, the author takes liberty to~omit its proof,
referring once more to the monography~\cite{picado-pultr12}.

\begin{thm}
  The (Reg) implies the (I-Haus).
\end{thm}
\begin{proof}
  Let $(a \wedge b, a \wedge b) \in U$.
  By the earlier lemma, we have $(x, y)\in U$ with $x \rb a$ and $y \rb b$.
  Recall the saturatedness from Remark~\ref{satur-def-eq} on
  page~\pageref{satur-def-eq}\thinspace:
  as $U$ is saturated, one gets
  \[
    \left\{ (x, y) \st x \rb a \right\} \subseteq U
    \Longrightarrow
    \left(\bigvee \{x \st x \rb a\}, y\right)\in U,
  \]
  or equally, using regularity,
  \[
    \left(a, y\right)\in U
  \]
  for all $y \rb b$.
  Wherefore, in the same manner: $(a, b) = \left(a, \bigvee \{y \st y \rb
  b\}\right)\in U$.
\end{proof}

\begin{cor}
  Regular locales are weakly Hausdorff.
\end{cor}
\begin{proof}
  Recall Theorem~\ref{IHaus->DSHaus} on page~\pageref{IHaus->DSHaus}\thinspace.
\end{proof}

It appears to be of great advantage to characterize (Reg) by a formula
that rather resembles the subfit definition:
\begin{lem} \label{reg-char}
  A locale is regular iff
  \[
    a \not\le b \qquad \Rightarrow \qquad \exists c: \quad a \vee c = 1 \quad
    \& \quad c^* \not\leq b
  \]
\end{lem}
\begin{proof}
  $\Rightarrow$:
  Suppose $a \not\le b$;
  that is, using the regularity, there is $x \rb a$ with $x \not\le b$ (otherwise
  $\bigvee \{x \st x \rb a\} \le b$).
  Let $c := x^*$.
  In other words, by $x \rb a$, one concludes $a \vee c = a \vee x^* = 1$.
  Additionally, $c^* \not\le b$: since otherwise from pseudocomplement property
  $x \le x^{**} = c^* \le b$, a contradiction.

  $\Leftarrow$:
  By~\ref{rb->leq} we always have the inequality $\bigvee \{x \st x \rb a\} \le
  a$.
  Hence, for contradiction assume $a \not\le \bigvee \{x \st x \rb a\}$.
  Then, there exists $c$ from the~premise.
  Specifically, $1 = a \vee c \le a \vee c^{**}$, which gives us $c^* \rb a$.
  Yet $c^* \not\le \bigvee \{x \st x \rb a\}$.
\end{proof}

The new definition of regularity is more suitable to prove
\begin{thm}
  The (Reg) implies the (Sfit).
\end{thm}
\begin{proof}
  Given $a \not\le b$, we acquire---from Lemma~\ref{reg-char}\thinspace---an
  element $c$ such that $a \vee c = 1$ and $c^* \not\le b$.
  What is more, it satisfies $b \vee c \ne 1$.
  If not so then
  \[
    b = b \vee 0 = b \vee (c \wedge c^*) = (b \vee c) \wedge (b \vee c^*) = 1
    \wedge (b \vee c^*) = b \vee c^*,
  \]
  the third identity from distributivity; subsequently, this contradicts $c^*
  \le b$.
\end{proof}

\begin{cor}
  The regularity also implies all the above-mentioned axioms of Hausdorff type.
\end{cor}
