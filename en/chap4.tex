\chapter{Regularly behaving regularity}

We are going to examine regularity in the pointless context;
how it is stronger than both strongly Hausdorff property and subfitness, and
thus, how it implies all the aforestated axioms.

\section{The $T_3$ axiom}

\begin{framed}
  \begin{df}[$T_3$]
    We will consider a topological space $X$ to be \emph{regular\/} (or
    satisfying the~\emph{$T_3$-axiom\/}) when
    \begin{center} \it
      for any $x\in X$ and any closed $A \not\owns x$ there are open $V_1\owns
      x$ and $V_2\supseteq A$ disjoint from~one another.
    \end{center}
  \end{df}
\end{framed}

Even though the definition mentions points, it may be elegantly reduced so as
not to do so.

\begin{prop} \label{reg-char}
  A space $X$ is regular iff each $U\in \Omega(X)$ can be depicted as
  \[
    U = \bigcup \{V\in \Omega(X) \st \overline{V} \subseteq U\}.
  \]
\end{prop}
\begin{proof}
  The inclusion~$\supseteq$ constantly holds (as $V\subseteq
  \overline{V}\subseteq U$).

  $\Rightarrow$:
  Choose any $x\in U$.
  Since $A := X\setminus U \not\owns x$ is closed, the existence
  of~$V_1$~and~$V_2$ from $(T_3)$ is stipulated.
  The disjointedness $V_1 \cap V_2 = \none$ leads to $V_1\subseteq X\setminus
  V_2$, which is, incidentally, a closed set.
  Therefore, $\overline{V_1}\subseteq X\setminus V_2$.

  Furthermore, the relation $A\subseteq V_2$ is undoubtedly equivalent to the
  relation $X\setminus V_2\subseteq X \setminus A = U$.
  Hence, for $V_x := V_1$ one has $\overline{V_x}\subseteq X\setminus V_2\subseteq
  U$.
  Such a~system $\{V_x \st x\in U\}$ constitutes a~subset cover of $U$, and
  consequently, finishes the proof of~the~inclusion $\subseteq$.

  $\Leftarrow$:
  With $U := X\setminus A$ take $V_x$ from above and set $V_1 := V_x$ and $V_2
  := X\setminus \overline{V_x}$.
\end{proof}

We may even ponder the possibility of eliminating the~closure in
the~description:

\begin{lem}
  $\overline{V}\subseteq U \quad \equiv \quad \exists W\in \Omega(X)\colon W \cap V =
  \none \;\; \& \;\; W \cup U = X$.
\end{lem}
\begin{proof}
  $\Rightarrow$:
  Take $W := X\setminus \overline{V}$.

  $\Leftarrow$:
  Recall closure's definition.
  Suppose $z\in \overline{V}\setminus U$.
  Since $W \cup U = X$, the $z$ must lie in the $W$;
  howbeit, the $W$ does not intersect the $V$---forcing $z\not\in
  \overline{V}$.
\end{proof}

Without loss of generality, the $W$ might be replaced by~the
pseudocomplement~$V^*$;
explicitly, the $X\setminus \overline{V}$ in $\Omega(X)$.%
\footnote{In frames, pseudocomplements are always present: $a^* := a
\rightarrow 0$ exist owing to frame's being Heyting algebra.}

Accordingly, for the sake of brevity, we introduce:
\begin{framed}
  \begin{nota}[$\rb$]
    \[
      V \rb U \quad \equiv \quad V^* \vee U = 1,
    \]
    which is referred to by pronouncing \emph{``V is rather below U''\/}.
  \end{nota}
\end{framed}

The next auxiliary lemma will be advantageous property for the rest of this
chapter.
\begin{lem}
  $a \rb b \Rightarrow a \leq b$.
\end{lem}
\begin{proof}
  With help of distributivity,
  \[
    a = a \wedge 1 = a \wedge (a^* \vee b) = (a \wedge a^*) \vee (a \wedge b) =
    0 \vee (a \wedge b) = a \wedge b;
  \]
  which is required to proved.
\end{proof}

Possessing $\rb$ notation, we are able to mimic characterization
in~\ref{reg-char} by
\begin{framed}
  \begin{df}[Reg]
    A locale is called \emph{regular\/} if and only if
    \[
      a = \bigvee \{x \st x \rb a\}
    \]
    for all its elements $a$.
  \end{df}
\end{framed}

Moreover, the proposition leads us to
\begin{cor}
  A topological space $X$ is regular iff its $\Omega(X)$ is regular.
\end{cor}

\section{Strength of regularity}

Now we are going to study the stated relationships between (Reg) and other
axioms.

\begin{lem} \label{oplus-vee-distrib}
  \[
    \left(\bigvee_{i\in J} a_i\right) \oplus \left(\bigvee_{i\in J} b_i\right)
    = \bigvee_{i\in J} \left(a_i \oplus b_i\right)
  \]
\end{lem}
\begin{proof}
  TBD
\end{proof}

\begin{lem}
  For a general locale $L$ and any of its saturated $U\in \left\uparrow
  d_L\right.$
  \[
    (a \wedge b, a \wedge b) \in U \qquad \Longrightarrow \qquad \forall x \rb
    a, \, y \rb b: \; (x, y) \in U
  \]
\end{lem}
\begin{proof}
  Beginning with indisputable $(x, y)\in x \oplus y$, we modify
  \begin{align*}
    x \oplus y &= (x \wedge (y^* \vee b)) \oplus (y \wedge (x^* \vee a)) \\
               &= (\; (x \wedge y^*) \; \vee \; (x \wedge b) \; ) \oplus (\; (y
    \wedge x^*) \; \vee \; (y \wedge a) \; )
  \end{align*}
  (equalities using ``rather-belowness'' and distributivity, respectively).

  Proceeding with Lemma~\ref{oplus-vee-distrib}\thinspace,
  \begin{align*}
     \ldots &= 
  \end{align*}
\end{proof}

The proof of the oncoming theorem will depend upon the reverse implication
of~\ref{meets-in-satur}\thinspace;
for conciseness of this thesis, the author takes liberty to omit its proof,
referring once more to the monography~\cite{picado-pultr12}.

\begin{thm}
  The (Reg) implies the (I-Haus).
\end{thm}
\begin{proof}
  Let $(a \wedge b, a \wedge b) \in U$.
  By the earlier lemma, $(x, y)\in U$ where $x \rb a$ and $y \rb b$.
  As $U$ is saturated (recall the definition from~\ref{satur-def-eq}),
  \[
    \left(\bigvee \{x \st x \rb a\}, y\right)\in U
  \]
  or equally, using regularity,
  \[
    \left(a, y\right)\in U
  \]
  for all $y \rb b$.
  Wherefore, in the same manner, $(a, b) = \left(a, \bigvee \{y \st y \rb
  b\}\right)\in U$.
\end{proof}
