\chapter{Complete regularity}

\section{The $T_{3\frac{1}{2}}$-axiom}

\begin{framed}
  \begin{df}[$T_{3\frac{1}{2}}$]
    A topological space $X$ is said to be \emph{completely regular\/} (or
    equally \emph{$T_{3\frac{1}{2}}$-space\/}) if
    \begin{center} \it
      for any $x\in X$ and any closed $A$ such that $A\not\owns x$ there is a
      continuous function $\phi\colon X \to \I$ with $\phi(x) = 0$ and $\phi[A]
      = \{1\}$.
    \end{center}
  \end{df}
\end{framed}

\begin{rem}
  The complete regularity implies regularity.
  (Take the continuous function~$\phi$ from the definition and set $V_1 :=
  \phi^{-1}[\langle 0, \frac{1}{2} \langle]$ and $V_2 := \phi^{-1}[\rangle
  \frac{1}{2}, 1 \rangle]$.)
\end{rem}

\section{Completely regular locales}

\begin{framed}
  \begin{nota}[$\cb$]
    For elements $a, b$ of a locale $L$ let us write
    \[
      a \cb b
    \]
    and say \emph{``a is completely below b''\/} if for every $r\in \langle 0,
    1 \rangle \cap \Q$ there exist $a_r\in L$ satisfying
    \begin{align*}
      a_0 &= a, \\
      a_1 &= b, \\
      \qquad a_p &\rb a_q \quad \text{ for } p < q
    \end{align*}
  \end{nota}
\end{framed}

\begin{rem} \label{cb-largest-interpol}
  Clearly, the relation \emph{interpolates\/}.
  On the other hand, for each interpolating subrelation $R\subseteq \, \rb \, $
  and any $a R b$ there are countably many elements lying (densily) between
  them.%
  \footnote{for instance, by~induction on~dyadic fractions}
  Thus, $R\subseteq \cb$, and therefore,
  \begin{center}
    \emph{$\cb$ is the largest interpolative subrelation of $\rb$.\/}
  \end{center}
\end{rem}

\begin{rem}
  By~\ref{rb->leq} on page~\pageref{rb->leq} $a \cb b$ implies $a \le b$.
\end{rem}

\begin{lem} \label{cb->continuous}
  Let $X$ be a~topological space and let $\{A_r\in \Omega(X) \st r\in \langle
  0, 1 \rangle \cap \Q\}$ be a~set witnessing $A \cb B$.
  Then function $f\colon X \to \I$ defined by
  \[
    f(x) := \inf \{r \st A_r \owns x\}
  \]
  is continuous.
\end{lem}
\begin{proof}
  Firstly,
  \[
    x\in f^{-1}[\langle 0, s \langle] \; \equiv \;
    f(x) < s \; \equiv \;
    \exists r < s, \; A_r \owns x \; \equiv \;
    x\in \bigcup \{ A_r \st r < s \},
  \]
  the second equivalence from a~standard infimum property in~linearly ordered
  sets. Secondly,
  \begin{align*}
    x\in f^{-1}[\rangle s, 1 \rangle] \; &\equiv \;
    f(x) > s \\
    &\equiv \; \exists r, r' \in \,\rangle s, f(x) \langle\,, \quad  r' > r
    \;\text{ and }\; x\not\in A_{r'} \;\text{ and }\; x\not\in \overline{A_r}
    \; \\
    &\equiv \; x\not\in \bigcap \left\{ \overline{A_r} \st r > s \right\}
    \equiv \; x\in \I\setminus \bigcap \left\{ \overline{A_r} \st r > s
    \right\},
  \end{align*}
  the latter equivalence by the density of~rational numbers, the fact that
  infimum is an~upper-bound and $A_r \rb A_{r'}$ (or equivalently
  $\overline{A_r}\subseteq A_{r'}$).

  To sum up, preimages
  \begin{align*}
    f^{-1}[\langle 0, s \langle] &= \bigcup \{ A_r \st r < s \}, \\
    f^{-1}[\rangle s, 1 \rangle] &= \I\setminus \bigcap \left\{ \overline{A_r} \st r > s \right\}
  \end{align*}
  of~$\I$'s subbasis are open;
  hence, $f$ is continuous.
\end{proof}

\begin{prop} \label{T3,5-char}
  A topological space $X$ is completely regular iff
  \[
    \forall U\in \Omega(X)\colon \quad U = \bigcup\{V\in \Omega(X) \st V \cb
    U\}
  \]
\end{prop}
\begin{proof}
  $\Rightarrow$:
  We will show the inclusion $\subseteq$. 
  Choose an~arbitrary $x\in U$.
  Take continuous $\phi$ separating $x$ from closed $A := X\setminus U$ and set
  \[
  V_r =
  \begin{cases}
    \phi^{-1}[\langle 0, \frac{r+1}{2} \langle] &\mbox{iff } r\in \langle 0, 1
    \langle \, \cap \, \Q  \\
    U                                   &\mbox{iff } r = 1.
  \end{cases}
  \]
  These open sets affirm $V_0 \cb U$.
  Indeed: 
  whenever $p < q$, we get
  \[
    \overline{V_p}
    = \overline{\phi^{-1}\left[ \left\langle 0, \frac{p+1}{2} \right\langle \right]}
    \subseteq \phi^{-1}\left[ \left\langle 0, \frac{p+1}{2} \right\rangle \right]
    \subseteq \phi^{-1}\left[ \left\langle 0, \frac{q+1}{2} \right\langle \right]
    = V_q,
  \]
  (the second inclusion since intervals $\left\langle 0, r \right\rangle$ are
  closed in $\I$).
  Further, $x\in V_0$ since $\phi(x) = 0$.
  Such sets $V_0$ (for every $x\in U$) cover all of the $U$; hence, the
  inclusion $\subseteq$ holds.

  $\Leftarrow$:
  For an open $U := X\setminus A$ and $x\in U$ there is a $V$ with $x\in V \cb U$.
  Thus, the $f$ from~\ref{cb->continuous} is the desired continuous function:
  first of all, $x\in V = V_0$ means $f(x) = 0$;
  furthermore, if $y\in A$ then $f(y) = \inf \none = 1$.
\end{proof}

\begin{framed}
  \begin{df}[CReg]
    A locale is called \emph{completely regular\/} if
    \[
      a = \bigvee \{x \st x \cb a\}
    \]
    for all its elements $a$.
  \end{df}
\end{framed}

Besides, from Proposition~\ref{T3,5-char} we gather that
\begin{center}
  \emph{a~space $X$ is completely regular iff $\Omega(X)$ is completely
  regular.\/}
\end{center}

\begin{rem}
  Because $\{x \st x \cb a\}\subseteq \{x \st x \rb a\}$, (CReg) implies (Reg).
\end{rem}
