\chapter{Normality}

\section{The $T_4$ axiom}

\begin{framed}
  \begin{df}[$T_4$]
    A topological space $X$ is \emph{normal\/} (or \emph{$T_4$-space\/}) if
    \begin{center} \it
      for every disjoint closed $A, B$ there exist disjoint open $U, V$ with
      $U\supseteq A, \, V\supseteq B$.
    \end{center}
  \end{df}
\end{framed}

The point-free counterpart is straightforward:

\begin{framed}
  \begin{df}[Norm]
    A locale $L$ is \emph{normal\/} if
    \[
      a \vee b = 1 \qquad \Rightarrow \qquad \exists u, v, \quad u \wedge v =
      0 \text{ and } a \vee u = 1 = b \vee v.
    \]
  \end{df}
\end{framed}

\begin{rem}
  As $u \wedge v = 0 \; \equiv \; v \le u^*$, we may define (Norm) equally by
  \[
    a \vee b = 1 \qquad \Rightarrow \qquad \exists u, \quad a \vee u = 1 = b
    \vee u^*.
  \]
\end{rem}

\medskip

Because the definition of $(T_4)$ is virtually point-free, we obtain following

\begin{cor}
  A topology $X$ is normal iff the locale $\Omega(X)$ is normal.
\end{cor}

\begin{lem}
  The relation $\rb$ interpolates in~normal locales.
\end{lem}
\begin{proof}
  Suppose $a$ is rather below $b$.
  That is, $a^* \vee b = 1$, and using the normality,\thinspace%
  \footnote{more precisely, the remark}
  we get $u\in L$ such that
  $a^* \vee u = 1 = b \vee u^*$.
  In other words, $a \rb u \rb b$.
\end{proof}

Combined with Remark~\ref{cb-largest-interpol} on
page~\pageref{cb-largest-interpol}\thinspace, we form

\begin{cor}
  (Norm) implies $\rb \, = \, \cb$;
  thence, in case of normal locales, regularity coincides with complete
  regularity.
\end{cor}

\begin{thm}
  (Norm) $\&$ (Sfit) $\Rightarrow$ (Reg).
\end{thm}
\begin{proof}
  Take $a\in L$ and put $b := \bigvee \{x \st x \rb a\}$.
  For contradiction, suppose that $a\not\le b$.
  Using the subfitness, there is $c\in L$ fulfilling $a \vee c = 1 \ne b \vee
  c$.
  Applying the normality, we find $u\in L$ with $a \vee u^* = 1 = c \vee u$.
  Particularly, $u \rb a$, which leads to~$u \le b$.
  Finally, $b \vee c \ge u \vee c = 1$; contradicting the subfitness.
\end{proof}
