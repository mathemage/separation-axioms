\chapter*{Preliminaries}
\addcontentsline{toc}{chapter}{Preliminaries}

\section*{Notations}

\begin{itemize}
\item $A \Subset B$ means {\sl set $A$ is a finite subset of set $B$\/}.
\item $\overline{S}$ 
\item $\Omega(X)$ denotes the lattice (or more specifically, the frame) of open
sets in $X$ together with $\subseteq$ as the ordering. 
By an abuse of notation, we denote the related underlying set in the very same
way.
\end{itemize}

\section*{Definitions}

\subsection*{Category theory}

\begin{itemize}
\item A \emph{category} is a
\item A \emph{product} of a system $\left(A_i\right)_{i\in J}$ is a
\end{itemize}

\subsection*{Point-set topology}

Some basic notions and facts about classical topology:

\begin{itemize}
\item A \emph{topological space} is a
\item A \emph{basis} of a topological space is a
\item Likewise, a \emph{subbasis} of a topological space is a
\end{itemize}

\begin{fact}
  The topological space $\left( \prod_{i\in J} X_i, \mathcal{S} \right)$, where
  $\mathcal{S} = \left\{ p_{i}^{-1}[U] \st i\in J, \; U\in \tau_i \right\}$
  denotes its subbasis, is the product of topological spaces $\left( X_i
  \right)_{i\in J}$;
  more precisely, the projections $\left( p_i: \left(\prod_{i\in J} X_i,
  \mathcal{S}\right) \to \left(X_i, \tau_i\right) \right)_{i\in J}$
  constitute the product in~the~category~{\bf Top}.
\end{fact}

\subsection*{Point-free topology}

\begin{itemize}
\item A \emph{frame} is a
\item A \emph{frame homomorphism} is a
The corresponding category will be called {\bf Frm}.
\item A \emph{locale} is a
\item A \emph{localic map} is a 
The corresponding category will be called {\bf Loc}.
\item Let $\bigoplus_{i\in J} L_i$ designate a \emph{product} in {\bf Loc} (and
\emph{coproduct} in {\bf Frm}); $L \oplus M$, $L_1 \oplus\cdots\oplus L_n$, etc.
for finite cases.
It may be demonstrated---though not straightaway vividly---that the product
coincides with
\[
  \textstyle\mathfrak{D}\left(\prod_{i\in J}' L_{i}\right)/R.
\]
For full details consult Chapter IV of Picado and Pultr \cite{picado-pultr12}.
\end{itemize}

\section*{Convention}

In the whole text we will omit the $T_0$ axiom, that is, we will assume all
topological spaces to be $T_0$-spaces.
Specifically, every topological space $(X, \tau)$ is considered to always
satisfy the condition
\begin{center} \it
  for any $x \ne y$ from $X$ there is an open set $U \in \tau$ such that $x \in U
  \not\owns y$ or $y \in U \not\owns x$.
\end{center}
