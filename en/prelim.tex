\chapter*{Preliminaries}
\addcontentsline{toc}{chapter}{Preliminaries}

\section*{Notations}

\begin{itemize}
\item $A \Subset B$ means {\sl set $A$ is a finite subset of set $B$\/}.
\item $\overline{S}$ 
\item $\Omega(X)$ denotes the lattice of open sets in $X$ together with
$\subseteq$ as the ordering. 
By an abuse of notation, we denote the related underlying set in the very same
way.
\end{itemize}

\section*{Definitions}

\subsection*{Point-set topology}

A \emph{topological space} is a

\subsection*{Point-free topology}

A \emph{frame} is a

\section*{Convention}

In the whole text we will omit the $T_0$ axiom, that is, we will assume all
topological spaces to be $T_0$-spaces.
Specifically, every topological space $(X, \tau)$ is considered to always
satisfy the condition
\begin{center} \it
  for any $x \ne y$ from $X$ there is an open set $U \in \tau$ such that $x \in U
  \not\owns y$ or $y \in U \not\owns x$.
\end{center}
