\chapter*{Preliminaries}
\addcontentsline{toc}{chapter}{Preliminaries}

\section*{Notations}

\begin{itemize}
\item $2^X$ is the set of~all subsets, explicitly, $2^X = \{ S \st S\subseteq X
\}$.

\item $A \Subset B$ means {\sl set $A$ is a finite subset of set $B$\/}.

\item $\Omega(X)$ denotes the lattice (or more specifically, the frame) of open
sets in $X$ together with $\subseteq$ as the ordering. 
By abuse of notation, we denote the related underlying set in the very same
way.

\item $\overline{S}$ is the~closure of $S$ in topology $(X, \Omega(X))$.
It is defined by
\[
  \overline{S} = \bigcap_{ \text{closed } A\colon \, A\supseteq S} A
  \, ,
\]
or equivalently,
\[
  \overline{S} = \{ x \in X \st U\in\Omega(X) \text{ and } U\owns x \Rightarrow
  U \cap S \ne \none \}.
\]
\item $f^*$ resp. $f_*$ is the~left resp. right adjoint of $f$ (viz ``Order
theory'').
\item $\left\downarrow U \right. = \bigcup_{u\in U} \{x \st x \le u\}$ and
$\left\uparrow U \right. = \bigcup_{u\in U} \{x \st x \ge u\}$.

\end{itemize}

\section*{Definitions}

\subsection*{Order theory}

\begin{itemize}
\item A \emph{pseudocomplement} of an element $x$ is $x^*$ such that $y \le x^*
\equiv x \wedge y = 0$.
\end{itemize}

\begin{fact}
  $x \le x^{**}$.
  (As the $x$, among others, meets with $x^*$ in $x^* \wedge x = 0$.)
\end{fact}

\begin{itemize}
\item A relation $R$ \emph{interpolates} if $R \subseteq R \circ R$.

\item Monotone maps $l\colon A \to B$ and $r\colon B \to A$ are \emph{Galois
  adjoints\/} ($l$ is a~\emph{left adjoint\/} of $r$; $r$ is a~\emph{right
  adjoint\/} of $l$) if for every $a\in A, \, b\in B$
  \[
    l(a) \le b \; \equiv a \le r(b).
  \]

\item A~set $U$ is a~\emph{down-set\/} resp. an~\emph{up-set\/} if
$\left\downarrow U \right. = U$ resp. $\left\uparrow U \right. = U$.
\end{itemize}

\subsection*{Category theory}

\begin{itemize}
\item A \emph{category} $\Ccal$ is a structure consisting of
  \begin{itemize}
  \item class~$\Ccal^O$ of \emph{objects\/}
  \item class~$||\Ccal||$ of \emph{morphisms\/} between objects 
  (a~concrete morphism $f$ between objects $A$ and $B$ is denoted by $f\colon A
   \to B$)
  \end{itemize}
further satisfying axioms of
  \begin{itemize}
  \item \emph{(identity)\/}
  For every $X\in \Ccal^O$ there is a~morphism $1_X$ such that for any $f\colon A \to B$
  we have $1_A \cdot f = f = f \cdot 1_B$.
  \item \emph{(asociativity)\/}
  Whenever $f\colon A \to B$, $g\colon B \to C$ and $h\colon C \to D$ then $h
  \cdot (g \cdot f) = (h \cdot g) \cdot f$.
  \end{itemize}

\item The \emph{opposite category}~$\mathcal{C}^{op}$ of a category
$\mathcal{C}$ comprises the same objects $\Ccal^O$ and ``reverse'' morphisms
$f^{op}\colon A \leftarrow B$ (for each morphism $f\colon A \to B$
in~$||\Ccal||$).

\item An \emph{epimorphism} is a right-cancellative morphism $e$, that is,
\[
  f_1 \cdot e = f_2 \cdot e \; \Longrightarrow \; f_1 = f_2.
\]
\end{itemize}

\begin{exmpl}
  In the category {\bf Loc} (see ``Point-free topology'') every localic map
  onto is epic.
\end{exmpl}

\begin{itemize}
\item A \emph{product} of a system $\left(A_i\right)_{i\in J}$ is a~system
of~\emph{projections\/}
\[
  \left(p_i\colon \prod_{j\in J} A_j \to A_i \right)_{i\in J}
\]
with the property that for arbitrary system $\left(f_i\colon X \to
A_i\right)_{i\in J}$ there is a~unique solution~$f\colon X \to \prod_{j\in
J} A_j$ to equations
\[
  p_i \cdot f = f_i \text{ for } i\in J.
\]

\item Similarly, a \emph{coproduct} is a product in the opposite category.
Namely, it is a~system of~\emph{injections\/}
\[
  \left(\iota_i\colon A_i \to \coprod_{j\in J} A_j \right)_{i\in J}
\]
such that every $\left(g_i\colon A_i \to X \right)_{i\in J}$ has a~single
solution~$g\colon \coprod_{j\in J} A_j \to X$ to~equations
\[
  g \cdot \iota_i = g_i \text{ for } i\in J.
\]

\item A \emph{diagonal} is (the only) morphism $\Delta\colon A \to \prod_{i\in
J} A$ solving the system $p_i\cdot \Delta = 1_A$ for $i \in J$.
Dually, a \emph{codiagonal} is the solution $\nabla\colon \coprod_{i\in J} A
\to A$ to the system $\nabla\cdot \iota_i = 1_A, \, i \in J$.
\end{itemize}


\subsection*{Point-set topology}

\begin{itemize}
\item A \emph{topological space} is a~structure of
  \begin{itemize}
  \item the set of \emph{points\/} $X$
  \item \emph{open sets\/} $\tau\subseteq 2^X$ holding axioms
    \begin{description}
    \item[(o1)] $\none, X \in \tau$
    \item[(o2)] $\left\{ U_i \st i\in J \right\} \subseteq \tau \; \Rightarrow
    \; \bigcup_{i\in J} U_i \in \tau$
    \item[(o3)] $U_1, U_2 \in \tau \; \Rightarrow \; U_1 \cap U_2 \in \tau$
    \end{description}
  \end{itemize}

\item The \emph{basis} of a topological space $(X, \tau)$ is the
$\mathcal{B}\subseteq \tau$ such that for every open $U$ we can write $U =
\bigcup \{V\in \mathcal{B} \st V\subseteq U \}$.

\item Likewise, $\mathcal{S}\subseteq \tau$ is a~\emph{subbasis} of a
topological space $(X, \tau)$ if $\left\{ \bigcap \mathcal{F} \st \mathcal{F}
\Subset \mathcal{S} \right\}$ is basis of~$\tau$.
\end{itemize}

\begin{exmpl}
  The topology $\I$ consists of underlying set $\langle 0, 1 \rangle$ and open
  sets generated by subbasis $\{ \langle 0, a \langle, \; \rangle a, 1 \rangle
  \st a\in \langle 0, 1 \rangle \}$.
\end{exmpl}

\begin{fact}
  The topological space $\left( \prod_{i\in J} X_i, \mathcal{S} \right)$, where
  $\mathcal{S} = \left\{ p_{i}^{-1}[U] \st i\in J, \; U\in \tau_i \right\}$
  denotes its subbasis, is the product of topological spaces $\left( X_i
  \right)_{i\in J}$;
  more precisely, the projections $\left( p_i\colon \left(\prod_{i\in J} X_i,
  \mathcal{S}\right) \to \left(X_i, \tau_i\right) \right)_{i\in J}$
  constitute the product in~the~category~{\bf Top}.
\end{fact}

\begin{itemize}
\item A function $f\colon X \to Y$ between topological space $X$ and $Y$ is
\emph{continuous} if 
\[
  U\in \Omega(Y) \; \Longrightarrow \; f^{-1}[U]\in \Omega(X).
\]
Since preimage preserves unions and finite intersections, this amounts to
requiring that $f^{-1}[U]$ is open for {\bf subbasic} $U$.
\end{itemize}

\subsection*{Point-free topology}

\begin{itemize}
\item A \emph{frame} is a~complete lattice $L$ satisfying ``infinite'' version
of distributivity
\[
  \bigvee A \wedge b = \bigvee \{ a \wedge b \st a\in A \}
\]
for every $A\subseteq L$ and $b\in L$.

\item A \emph{frame homomorphism} (between frames $M$ and $L$) is a~map
$h\colon M \to L$ preserving all joins (in particular, $h(0) = 0$) and~all
finite meets (especially, $h(1) = 1$).

\item The category of~frames and frame homomorphisms will be denoted by {\bf
Frm}.

\item A \emph{locale} is a~frame (it has different name because of~different
subsequent morphisms).

\item A \emph{localic map} (between locales $L$ and $M$) is a mapping $f\colon
L \to M$ having left (Galois) adjoints $f^*\colon M \to L$, which is also a
frame homomorphism.
Hence, localic maps are infima-preserving $f$ and their left adjoints $f^*$
preserves finite meets.

\item We will write {\bf Loc} for the category of~locales and localic maps.
Thus, $\mathbf{Loc} = \mathbf{Frm}^{op}$.

\item A \emph{sublocale} of locale $L$ is a $S\subseteq L$ satisfying
  \begin{description}
  \item[(S1)] $M\subseteq S \Longrightarrow \bigwedge M\in S$
  \item[(S2)] $x\in L \text{ and } s\in S \Longrightarrow x \rightarrow s \in S$ 
  \end{description}
\end{itemize}

\begin{fact}
  The set-theoretic image $f[L]$ under a localic map $f\colon L\to M$
  is a~sublocale of~$M$.
\end{fact}

\subsubsection*{On coproducts in Frm}

In the category of~frames we have a~coproduct
\[
  \iota_i: L_i \to L_1 \oplus L_2, \, i = 1, 2
\]
(there is a general coproduct but we will need a coproduct of two objects
 only).
It can be constructed as follows.

\begin{itemize}
\item A down-set $U\subseteq L_1 \times L_2$ is \emph{saturated\/} if for all
systems $(x_i)_{i\in J}$ in $L_1$ resp. $L_2$ and all $y$ in $L_2$ resp. $L_1$
we have \label{df:satur}
  \begin{align*}
    \{ (x_i, y) \st i\in J \} \subseteq U \, &\Rightarrow \, (\bigvee x_i, y)\in U,
    \text{ and} \\
    \{ (y, x_i) \st i\in J \} \subseteq U \, &\Rightarrow \, (y, \bigvee x_i)\in U.
  \end{align*}

\item $L_1 \oplus L_2$ will be the set of~all saturated downsets in $L_1
\times L_2$.

\item The definition above concerns also the void $J$, and hence, for each
saturated down-set~$U$ we get
\[
  U\supseteq \n := \{ (x, y) \st x = 0 \text{ or } y = 0 \}.
\]
Since $\n$ itself is obviously saturated, it is \emph{the least saturated
down-set\/}.

\item For $(a, b)\in L_1 \times L_2$ the down-set
\[
  \darr (a, b) \cup \n
\]
is saturated.
This element of $L_1 \oplus L_2$ is denoted by
\[
  a \oplus b
\]
and we have the coproduct injections $\iota_i: L_i \to L_1 \oplus L_2$ defined
by
\[
  \iota_1(a) := a \oplus 1, \; \iota_2(b) := 1 \oplus b.
\]

\begin{cor*}
  Due to the coordinatewise meet, we have
  \[
    a \oplus b = \iota_1(a) \wedge \iota_2(b).
  \]
\end{cor*}

\item Note that for all saturated $U$ we have
\[
  U
  = \bigcup \{ (a, b) \st (a, b)\in U \}
  = \bigcup \{ a \oplus b \st (a, b)\in U \}
\]
(the second equality because $U$ is a~down-set).
As the set-theoretic union of down-sets is also a~down-set, it coincides with
their join.
Therefore, we may write 
\[
  U = \bigvee \{ a \oplus b \st (a, b)\in U \},
\]
and thus, \emph{the elements $a \oplus b$ generate $L_1 \oplus L_2$\/}.
\end{itemize}

For full details consult Chapter~IV of Picado and Pultr \cite{picado-pultr12}.

We have the following facts \ldots

\clearpage

\begin{itemize}
\item A \emph{closed sublocale} $S$ is a \ldots
Besides, the \emph{closure} of $S$ is the least closed sublocale containing $S$.
Since \ldots On the other hand, \ldots
\end{itemize}

\begin{exmpl} \label{(co)diag-in-Loc}
  The diagonal localic map $\Delta\colon L \to L \oplus L$ in {\bf Loc} is
  given by formula
  \[
    \Delta(a) = \left\{ (x, y) \st x \wedge y \leq a \right\},
  \]
  and likewise, its left adjoint (the codiagonal $\nabla\colon L \oplus L \to
  L$) is described by
  \[
    \nabla(U) = \bigvee \left\{ x \st (x, x)\in U \right\},
  \]
  as it is more explained in Corollary~IV.5.5.2 and Section~IV.5.2 of~Picado
  and Pultr \cite{picado-pultr12}.
\end{exmpl}

\section*{Convention}

In the whole text we will neglect the $T_0$ axiom, that is, we will assume all
topological spaces to be $T_0$-spaces.
Thus, every topological space $(X, \tau)$ satisfies the condition
\begin{center} \it
  for any $x \ne y$ from $X$ there is an open set $U \in \tau$ such that $x \in U
  \not\owns y$ or $y \in U \not\owns x$.
\end{center}
