\chapter*{Preliminaries}
\addcontentsline{toc}{chapter}{Preliminaries}

\section*{Notations}

\begin{itemize}
\item $2^X$ is the set of~all subsets, explicitly, $2^X = \{ S \st S\subseteq X
\}$.

\item $A \Subset B$ means {\sl set $A$ is a finite subset of set $B$\/}.

\item $\Omega(X)$ denotes the lattice (or more specifically, the frame) of open
sets in $X$ together with $\subseteq$ as the ordering. 
By abuse of notation, we denote the related underlying set in the very same
way.

\item $\overline{S}$ is the~closure of $S$ in topology $(X, \Omega(X))$ defined
by
\[
  \overline{S} = \bigcap_{ \substack{
    A: A\subseteq S, \\
   \text{A is closed}}}
   A \, ,
\]
or equivalently,
\[
  \overline{S} = \{ x \in X \st U\in\Omega(X) \text{ and } U\owns x \Rightarrow
  U \cap S \ne \none \}.
\]
\end{itemize}

\section*{Definitions}

\subsection*{Order theory}

\begin{itemize}
\item A \emph{pseudocomplement} of an element $x$ is $x^*$ such that $y \le x^*
\equiv x \wedge y = 0$.
  \begin{fact}
    $x \le x^{**}$.
    (As the $x$, among others, meets with $x^*$ in $x^* \wedge x = 0$.)
  \end{fact}

\item A relation $R$ \emph{interpolates} if $R \subseteq R \circ R$.
\end{itemize}

\subsection*{Category theory}

\begin{itemize}
\item A \emph{category} $\Ccal$ is a structure consisting of
  \begin{itemize}
  \item class~$\Ccal^O$ of \emph{objects\/}
  \item class~$||\Ccal||$ of \emph{morphisms\/} between objects 
  (a~concrete morphism $f$ between objects $A$ and $B$ is denoted by $f\colon A
   \to B$)
  \end{itemize}
further satisfying axioms of
  \begin{itemize}
  \item \emph{(identity)\/}
  For every $X\in \Ccal^O$ there is a~morphism $1_X$ such that for any $f\colon A \to B$
  we have $1_A \circ f = f = f \circ 1_B$.
  \item \emph{(asociativity)\/}
  Whenever $f\colon A \to B$, $g\colon B \to C$ and $h\colon C \to D$ then $h
  \circ (g \circ f) = (h \circ g) \circ f$.
  \end{itemize}

\item The \emph{opposite category}~$\mathcal{C}^{op}$ of a category
$\mathcal{C}$ comprises the same objects $\Ccal^O$ and ``reverse'' morphisms
$f^{op}\colon A \leftarrow B$ for each morphisms $f\in ||\Ccal||$.

\item An \emph{epimorphism} is a right-cancellative morphism $e$, that is,
\[
  f_1 \circ e = f_2 \circ e \; \Longrightarrow \; f_1 = f_2.
\]
\end{itemize}

\begin{exmpl}
  In the category {\bf Loc} (see ``Point-free topology'') every localic map
  onto is epimorphic.
\end{exmpl}

\begin{itemize}
\item A \emph{product} of a system $\left(A_i\right)_{i\in J}$ is a~system
\[
  \left(p_i\colon \prod_{j\in J} A_j \to A_i \right)_{i\in J}
\]
with property that for arbitrary system $\left(f_i\colon X \to A_i\right)_{i\in
J}$ there is a~unique solution~$f\colon X \to \prod_{j\in J} A_j$ to equations
\[
  p_i \circ f = f_i \text{ for } i\in J.
\]
Similarly, a \emph{coproduct} is a product in the opposite category.

\item A \emph{diagonal} is (the only) morphism $\Delta\colon A \to \prod_{i\in
J} A$ solving the system $p_i\cdot \Delta = id$ for $i \in J$.
\end{itemize}


\subsection*{Point-set topology}

Some basic notions and facts about classical topology:

\begin{itemize}
\item A \emph{topological space} is a~structure of
  \begin{itemize}
  \item the set of \emph{points\/} $X$
  \item \emph{open sets\/} $\tau\subseteq 2^X$ holding axioms
    \begin{description}
    \item[(ot1)] $\none, X \in \tau$
    \item[(ot2)] $\left\{ U_i \st i\in J \right\} \subseteq \tau \; \Rightarrow
    \; \bigcup_{i\in J} U_i \in \tau$
    \item[(ot3)] $U_1, U_2 \in \tau \; \Rightarrow \; U_1 \cap U_2 \in \tau$
    \end{description}
  \end{itemize}

\item The \emph{basis} of a topological space $(X, \tau)$ is the
$\mathcal{B}\subseteq \tau$ such that for every open $U$ we can write $U =
\bigcup \{V\in \mathcal{B} \st V\subseteq U \}$.

\item Likewise, $\mathcal{S}\subseteq \tau$ is a~\emph{subbasis} of a
topological space $(X, \tau)$ if $\left\{ \bigcap \mathcal{F} \st \mathcal{F}
\Subset \mathcal{S} \right\}$ is basis of~$\tau$.
\end{itemize}

\begin{exmpl}
  The topology $\I$ consists of underlying set $\langle 0, 1 \rangle$ and open
  sets generated by subbasis $\{ \langle 0, a \langle, \; \rangle a, 1 \rangle
  \st a\in \langle 0, 1 \rangle \}$.
\end{exmpl}

\begin{fact}
  The topological space $\left( \prod_{i\in J} X_i, \mathcal{S} \right)$, where
  $\mathcal{S} = \left\{ p_{i}^{-1}[U] \st i\in J, \; U\in \tau_i \right\}$
  denotes its subbasis, is the product of topological spaces $\left( X_i
  \right)_{i\in J}$;
  more precisely, the projections $\left( p_i\colon \left(\prod_{i\in J} X_i,
  \mathcal{S}\right) \to \left(X_i, \tau_i\right) \right)_{i\in J}$
  constitute the product in~the~category~{\bf Top}.
\end{fact}

\begin{itemize}
\item A function $f\colon X \to Y$ between topological space $X$ and $Y$ is
\emph{continuous} if 
\[
  U\in \Omega(Y) \; \Longrightarrow \; f^{-1}[U]\in \Omega(X).
\]
Since preimage preserves unions and finite intersections, this amounts to
requiring that $f^{-1}[U]$ is open for {\bf subbasic} $U$.
\end{itemize}

\subsection*{Point-free topology}

\begin{itemize}
\item A \emph{frame} is a
\item A \emph{frame homomorphism} is a
The corresponding category will be called {\bf Frm}.
\item A \emph{locale} is a
\item A \emph{localic map} is a \ldots
We will write {\bf Loc} for the category of~locales and localic maps.
\item A \emph{sublocale} $S$ is a \ldots
  \begin{fact}
    The set-theoretic image $f[L]$ under a localic map $f\colon L\to M$
    is a~sublocale of~$M$.
  \end{fact}
\item A \emph{closed sublocale} $S$ is a \ldots
Besides, the \emph{closure} of $S$ is the least closed sublocale containing $S$.
Since \ldots On the other hand, \ldots
\item Given a relation $R\subseteq L \times L$, an element $s\in L$ is
\emph{$R$-saturated} (shortly \emph{saturated} whenever the context is clear)
  iff
\[
  \forall a, b, c\in L: \; aRb \quad \Rightarrow \quad \left( a \wedge c \leq s
  \, \Leftrightarrow \, b \wedge c \leq s \right)
\]
\item The letter $\n$ is the notation for the smallest saturated\ldots
\item Let $\bigoplus_{i\in J} L_i$ designate a \emph{product} in {\bf Loc} (and
\emph{coproduct} in {\bf Frm}); $L \oplus M$, $L_1 \oplus\cdots\oplus L_n$, etc.
for finite cases.
It may be demonstrated---though not straightaway vividly---that the product
coincides with
\[
  \textstyle\mathfrak{D}\left(\prod_{i\in J}\nolimits' L_{i}\right)/R.
\]
For full details consult Chapter~IV of Picado and Pultr \cite{picado-pultr12}.
\item Let $a \oplus b \in L \oplus L$ with $a, b\in L$ stand for
$\left\downarrow (a, b) \right. \cup \n$---the least saturated down-set
containing $(a, b)$.
\end{itemize}

\begin{exmpl} \label{(co)diag-in-Loc}
  The diagonal localic map $\Delta\colon L \to L \oplus L$ in {\bf Loc} is
  given by formula
  \[
    \Delta(a) = \left\{ (x, y) \st x \wedge y \leq a \right\},
  \]
  and likewise, its left adjoint (the codiagonal $\nabla\colon L \oplus L \to
  L$) is described by
  \[
    \nabla(U) = \bigvee \left\{ x \st (x, x)\in U \right\},
  \]
  as it is more explained in Corollary~IV.5.5.2 and Section~IV.5.2 of~Picado
  and Pultr \cite{picado-pultr12}.
\end{exmpl}

\section*{Convention}

In the whole text we will neglect the $T_0$ axiom, that is, we will assume all
topological spaces to be $T_0$-spaces.
Thus, every topological space $(X, \tau)$ satisfies the condition
\begin{center} \it
  for any $x \ne y$ from $X$ there is an open set $U \in \tau$ such that $x \in U
  \not\owns y$ or $y \in U \not\owns x$.
\end{center}
