\chapter*{Preliminaries}
\addcontentsline{toc}{chapter}{Preliminaries}

\section*{Notations}

\begin{itemize}
\item $A \Subset B$ means {\sl set $A$ is a finite subset of set $B$\/}.
\item $\overline{S}$ 
\item $\Omega(X)$ denotes the lattice (or more specifically, the frame) of open
sets in $X$ together with $\subseteq$ as the ordering. 
By an abuse of notation, we denote the related underlying set in the very same
way.
\end{itemize}

\section*{Definitions}

\subsection*{Order theory}

\begin{itemize}
\item A \emph{pseudocomplement} is \ldots
\end{itemize}

\subsection*{Category theory}

\begin{itemize}
\item A \emph{category} is a collection of objects and morphisms \ldots
\item The \emph{opposite category}~$\mathcal{C}^{op}$ of a category
$\mathcal{C}$ is the category \ldots
\item An \emph{epimorphism} is a right-cancellative morphism meaning\ldots
In the category {\bf Loc} every localic map onto is, in addition, an
epimorphism.
\item A \emph{product} of a system $\left(A_i\right)_{i\in J}$ is a \ldots
Similarly, a \emph{coproduct} is a product in the opposite category.
\item A \emph{diagonal} is (the only) morphism $\Delta\colon A \to \prod_{i\in
J} A$ solving the system $p_i\cdot \Delta = id$ for $i \in J$.
  \begin{exmpl} \label{(co)diag-in-Loc}
    The diagonal localic map $\Delta\colon L \to L \oplus L$ in {\bf Loc} is
    given by formula
    \[
      \Delta(a) = \left\{ (x, y) \st x \wedge y \leq a \right\},
    \]
    and likewise, its left adjoint (the codiagonal $\nabla\colon L \oplus L \to
    L$) is described by
    \[
      \nabla(U) = \bigvee \left\{ x \st (x, x)\in U \right\},
    \]
    as it is more explained in Corollary~IV.5.5.2 and Section~IV.5.2 of~Picado
    and Pultr \cite{picado-pultr12}.
  \end{exmpl}
\end{itemize}


\subsection*{Point-set topology}

Some basic notions and facts about classical topology:

\begin{itemize}
\item A \emph{topological space} is a
\item A \emph{basis} of a topological space is a
\item Likewise, a \emph{subbasis} of a topological space is a
\end{itemize}

\begin{fact}
  The topological space $\left( \prod_{i\in J} X_i, \mathcal{S} \right)$, where
  $\mathcal{S} = \left\{ p_{i}^{-1}[U] \st i\in J, \; U\in \tau_i \right\}$
  denotes its subbasis, is the product of topological spaces $\left( X_i
  \right)_{i\in J}$;
  more precisely, the projections $\left( p_i\colon \left(\prod_{i\in J} X_i,
  \mathcal{S}\right) \to \left(X_i, \tau_i\right) \right)_{i\in J}$
  constitute the product in~the~category~{\bf Top}.
\end{fact}

\subsection*{Point-free topology}

\begin{itemize}
\item A \emph{frame} is a
\item A \emph{frame homomorphism} is a
The corresponding category will be called {\bf Frm}.
\item A \emph{locale} is a
\item A \emph{localic map} is a \ldots
We will write {\bf Loc} for the category of~locales and localic maps.
\item A \emph{sublocale} $S$ is a \ldots
  \begin{fact}
    The set-theoretic image $f[L]$ under a localic map $f\colon L\to M$
    is a~sublocale of~$M$.
  \end{fact}
\item A \emph{closed sublocale} $S$ is a \ldots
Besides, the \emph{closure} of $S$ is the least closed sublocale containing $S$.
Since \ldots On the other hand, \ldots
\item Given a relation $R\subseteq L \times L$, an element $s\in L$ is
\emph{$R$-saturated} (shortly \emph{saturated} whenever the context is clear)
  iff
\[
  \forall a, b, c\in L: \; aRb \quad \Rightarrow \quad \left( a \wedge c \leq s
  \, \Leftrightarrow \, b \wedge c \leq s \right)
\]
\item The letter $\n$ is the notation for the smallest saturated\ldots
\item Let $\bigoplus_{i\in J} L_i$ designate a \emph{product} in {\bf Loc} (and
\emph{coproduct} in {\bf Frm}); $L \oplus M$, $L_1 \oplus\cdots\oplus L_n$, etc.
for finite cases.
It may be demonstrated---though not straightaway vividly---that the product
coincides with
\[
  \textstyle\mathfrak{D}\left(\prod_{i\in J}\nolimits' L_{i}\right)/R.
\]
For full details consult Chapter~IV of Picado and Pultr \cite{picado-pultr12}.
\item Let $a \oplus b \in L \oplus L$ with $a, b\in L$ stand for
$\left\downarrow (a, b) \right. \cup \n$---the least saturated down-set
containing $(a, b)$.
\end{itemize}

\section*{Convention}

In the whole text we will omit the $T_0$ axiom, that is, we will assume all
topological spaces to be $T_0$-spaces.
Accurately, every topological space $(X, \tau)$ is considered to always satisfy
the condition
\begin{center} \it
  for any $x \ne y$ from $X$ there is an open set $U \in \tau$ such that $x \in U
  \not\owns y$ or $y \in U \not\owns x$.
\end{center}
