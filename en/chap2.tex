\chapter{Weakly and strongly Hausdorff}

In this chapter we will consider point-free analogies of the $T_2$
axiom---a~separation not quite easy to imitate in frames.

\section{The $T_2$ axiom}

\begin{framed}
  \begin{df}[$T_2$]
    A topological space $(X, \tau)$ is called \emph{Hausdorff\/} (or
    a~\emph{$T_2$-space\/}) if
    \begin{center} \it
      for any $x \ne y$ from~$X$ there are disjoint $U$ and $V$ with $U\owns x$
      and $V\owns y$.
    \end{center}
  \end{df}
\end{framed}

\begin{rem} \label{T2->T1}
  Trivially, every $T_2$-space is $T_1$.
\end{rem}

\section{The Dowker-Strauss approach}

In 1972 Dowker and Strauss \cite{ds72} introduced the following condition:

\begin{framed}
  \begin{df}
    A locale $L$ satisfies the $S_2'$ axiom if
    \[
      (S_2') \qquad
      a, b \ne 1 \text{ and } a \vee b = 1 \quad \Rightarrow \quad \exists
      u\not\leq a, v\not\leq b, \quad u \wedge v = 0.
    \]
  \end{df}
\end{framed}

\begin{prop} \label{Haus->S2'}
  For every Hausdorff space $X$ the corresponding locale $\Omega(X)$ satisfies
  $S_2'$.
\end{prop}

\begin{proof}
  Open sets $A, B \ne X$ with $A \cup B = X$ contain $x\in X\setminus A =
  B\setminus A$ and $y\in X\setminus B = A\setminus B$.
  Particularly, we observe $x \ne y$ and, on top of that, receive the disjoint
  $U$ and $V$ from the $T_2$-definition.
  Evidently, $U\not\subseteq A$ (because of $x$) and $V\not\subseteq B$
  (because of $y$).
\end{proof}

\begin{rem}
  The reverse implication requires, in addition, the subfitness.
  For the~proof see Proposition 7 of~Dowker and Strauss
  \cite{ds72}.
\end{rem}

The $S_2'$ axiom is often replaced by a~somewhat stronger condition:

\begin{framed}
  \begin{df}[DS-Haus]
    A locale $L$ is considered to be \emph{weakly Hausdorff\/} (also
    \emph{DS-Hausdorff} as in \emph{Dowker-Straus-Hausdorff}) if
    \[
      a \vee b \not\in \left\{a, b\right\} \qquad \Rightarrow \qquad \exists
      u\not\leq a, v\not\leq b, \quad u \wedge v = 0.
    \]
  \end{df}
\end{framed}

\section{Isbell's approach}

Here is a~characterization of~the classical~$T_2$:

\begin{prop}
  A topological space $X$ is Hausdorff space if and only if the~diagonal
  $\Delta = \left\{(x, x) \st x\in X \right\}$ is closed in the product
  $X\times X$.
\end{prop}

\begin{proof}
  $\Rightarrow:$ Any element $(x, y)\not\in \Delta$, that means $x \ne y$, is
  separable from $\Delta$:
  namely, by~the~open $p_1^{-1}[U] \cap p_2^{-1}[V]$ with non-intersecting $U$
  and $V$ from the definition of~$(T_2)$.
  Hence, $(x, y)\not\in \overline{\Delta}$ concludes to $\Delta =
  \overline{\Delta}$, which is closed.

  $\Leftarrow:$ Choose $x \ne y$, in other words, $(x, y)\not\in \Delta$.
  Then $(x, y)$ has an open basic neighbourhood $U_1\times U_2$ disjoint from
  $\Delta$.
  That is, $U_1 \cap U_2 = \none$.
\end{proof}

This was used by Isbell \cite{isbell72} for point-free counterpart of the $T_2$
axiom:
Hausdorff locales are defined as those in which the diagonal (see
Example~\ref{(co)diag-in-Loc})
\[
  \Delta\colon L \to L \oplus L
\]
is a closed localic map.
Namely, the image~$\Delta[L]$ is a closed sublocale in~$L \oplus L$.

\begin{lem}
  $\bigwedge \Delta[L] = \Delta(0_L)$.
\end{lem}

\begin{proof}
  Again, recall Example~\ref{(co)diag-in-Loc}\thinspace.

  $\subseteq$: The down-set $\Delta(0_L)$ is among elements of $\Delta[L]$; the
  infimum lies below all of~those.

  $\supseteq$: By the formula of the diagonal morphism,
  \[
    \Delta(0_L)
    = \left\{(x, y) \st x \wedge y \leq 0_L \right\}
    \subseteq \left\{(x, y) \st x \wedge y \leq a \right\}
    = \Delta(a)
  \]
  for every $a\in L$.
  From the property of the~infimum, $\Delta(0_L) \subseteq \bigwedge
  \Delta[L]$.
\end{proof}

Combining the above lemma with the~formula of~the sublocalic closure
\[
  \overline{S} = \left\uparrow \bigwedge \right. S,
\]
Isbell's condition reduces to
\[
  \Delta[L] = \left\uparrow \Delta(0_L),\right.
\]
which---after denoting the crucial element $\Delta(0_L)$ by $d_L$---produces
the final

\begin{framed}
  \begin{df}[I-Haus]
    A locale $L$ is called \emph{strongly Hausdorff\/} (equally
    \emph{I-Hausdorff}: short for \emph{Isbell-Hausdorff}) iff
    \[
      \Delta[L] = \uparr d_L
    \]
  \end{df}
\end{framed}

\begin{thm} \label{IHaus->DSHaus}
  (I-Haus) implies (DS-Haus).
\end{thm}

\begin{lem}
  Suppose a join-basis $\mathcal{B}$ of a locale $L$ and a relation $R
  \subseteq L \times L$ possessing property
  \[
    \forall a, b\in L \; \forall c \in \mathcal{B}: \quad aRb \Rightarrow (a
    \wedge c)R(b \wedge c).
  \]
  Such a premise cause an element $s\in L$ to be $R$-saturated if and only if
  \[
    aRb \qquad \Rightarrow \qquad (a \leq s \; \Leftrightarrow \; b \leq s).
  \]
\end{lem}
\begin{proof}
  $\Rightarrow$: Trivial (put $c := 1$ in the definition of saturatedness).

  $\Leftarrow$: For a general $c\in L$ there are $c_i\in \mathcal{B}$ holding
  $c = \bigvee_{i\in J} c_i$.
  Following from frame distributivity, the~definition of~$\bigvee$ and the
  special property of~$R$, we get
  \begin{align*}
    a \wedge c \leq s \quad
    &\equiv a \wedge \bigvee_{i\in J} c_i \leq s \\
    &\equiv \bigvee_{i\in J} (a \wedge c_i) \leq s \\
    &\equiv \forall i\in J: a \wedge c_i \leq s \\
    &\equiv \forall i\in J: b \wedge c_i \leq s \\
    &\equiv \bigvee_{i\in J} (b \wedge c_i) \leq s \\
    &\equiv b \wedge \bigvee_{i\in J} c_i \leq s
    \equiv \quad b \wedge c \leq s;
  \end{align*}
  quite easily done.
\end{proof}

\begin{rem} \label{satur-def-eq}
  If additionally $aRb \Rightarrow a \leq b$, the previous alternative can be
  rewritten to
  \[
    aRb \quad \Rightarrow \quad (a \leq s \quad \Rightarrow \quad b \leq s)
  \]
  or even simpler
  \[
    aRb \; \& \; a \leq s \quad \Longrightarrow \quad b \leq s.
  \]

  Recall the products in {\bf Loc} (viz Preliminaries).
  The relation $R$ used in its definition satisfy both aforementioned
  conditions;
  hence, in case of binary products, $R$-saturated sets are precisely the
  down-sets $U \subseteq L_1 \times L_2$ fulfilling 
  \[
    \textstyle
    \{ (x_i, y) \st i \in J \} \subseteq U \quad \Longrightarrow \quad
    \left(\bigvee_{i\in J} x_i, y\right)\in U
  \]
  and
  \[
    \textstyle
    \{ (x, y_i) \st i \in J \} \subseteq U \quad \Longrightarrow \quad \left(x,
    \bigvee_{i\in J} y_i\right)\in U.
  \]
  Again, for further specification have a look at Picado and Pultr
  \cite{picado-pultr12}.
\end{rem}

\begin{lem} \label{downsets-satur}
  The down-set
  \[
    U = \left\downarrow(a, a \wedge b)\right. \cup \left\downarrow(a \wedge b, b)\right. \cup \n
  \]
  is saturated in $L \times L$ for arbitrary $a, b\in L$.
\end{lem}
\begin{proof}
  We are going to verify the saturation from~\ref{satur-def-eq}\thinspace.
  First of all, let us have $(x_i, y)\in U$ for $i\in J$.

  \underline{Case $y = 0$}:
  obviously, $(\bigvee_{i\in J} x_i, y) = (\bigvee_{i\in J} x_i, 0)\in \n
  \subseteq U$.

  \underline{Case $y \ne 0$ and $y \leq a \wedge b$}:
  then $x_i \leq a$ anyway, that is, $(\bigvee_{i\in J} x_i, y) \in
  \left\downarrow(a, a \wedge b)\right.$, which is a subset of $U$.

  \underline{Case $y \ne 0$ and $y \not\leq a \wedge b$}:
  in this case, inevitably, $y \leq b$ as well as $x_i \leq a \wedge b$; and
  analogously, again $(\bigvee_{i\in J} x_i, y) \in \left\downarrow(a \wedge b,
  b)\right. \subseteq U$.

  The $(x, \bigvee_{i\in J} y_i)$ by symmetry.
\end{proof}

\begin{lem} \label{meets-in-satur}
  In case of a strongly Hausdorff locale $L$, we have
  \[
    \Delta(\nabla(U)) = U
  \]
  for any saturated $U \supseteq d_L$.
\end{lem}
\begin{proof}
  By (I-Haus), every saturated $U\in \left\uparrow d_L \right. = \Delta[L]$ is an image
  $\Delta(a)$ for some $a\in L$.
  Let us write $\delta(U)$ for the mentioned $a$.
  In other words, $\Delta\delta(U) = U$.

  Remind yourselves of~Example~\ref{(co)diag-in-Loc}\thinspace.
  Notice $\Delta^*(a \oplus a) = a$;
  hence, the codiagonal~$\nabla$ is onto, and consequently, an epimorphism.
  Furthemore, using a standard property of~Galois adjoints $\nabla \Delta
  \nabla = \nabla$,
  we get
  \[
    \nabla \Delta = id.
  \]

  Joining the two observations results in
  \[
    \nabla (U) = \nabla (\Delta\delta (U)) = (\nabla \Delta)\delta (U) = \delta(U),
  \]
  which produce the desired $U = \Delta \delta (U) = \Delta \nabla (U)$.
\end{proof}

\begin{lem} \label{meets-in-satur}
  Let $L$ be an I-Hausdorff locale.
  Then, one obtains the~implication
  \[
    \left( a \wedge b, a \wedge b \right) \in U
    \; \Rightarrow \;
    \left( a, b \right) \in U
  \]
  for all saturated $U \supseteq d_L$.
\end{lem}
\begin{proof}
  Whenever $(a \wedge b, a \wedge b)\in U$, we see from the formula for
  $\nabla$ that
  \[
    a \wedge b \leq \bigvee \left\{ x \st (x, x) \in U \right\} = \nabla(U).
  \]
  Having in mind the formula for $\Delta$, immediately $(a, b) \in \Delta( \nabla(U) ) = U$
  (the final equality by the previous lemma).
\end{proof}

\begin{rem}
  The last and the penultimate implication actually hold in forms
  of~equivalences.
  Nevertheless, the reverse implications will be spared since we will make no
  use of them.
  A curious reader may survey Section~2 of Chapter~V in~Picado
  and~Pultr~\cite{picado-pultr12}.
\end{rem}

Now we can prove the theorem.
In fact, we are going to imply a~stronger version of the (DS-Haus):
\[
  a \vee b \not\in \left\{a, b\right\} \qquad \Rightarrow \qquad \exists
  u\not\leq b, v\not\leq a: \quad u \wedge v = 0, \quad \boxed{\left(u,
  v\right) \leq \left(a, b\right)}
\]

\begin{proof}[Proof of~{\bf \ref{IHaus->DSHaus}}\thinspace]
  For a contradiction: let there be an I-Hausdorff $L$, which is not weakly
  Hausdorff.
  That means, we have $a, b$ with $a \not\leq b, \, b \not\leq a$ and such that
  \[
    u \wedge v = 0 \quad \& \quad \left(u, v\right) \leq \left(a, b\right)
    \qquad \Longrightarrow \qquad
    u \leq b \; \textrm{ or } \; v \leq a.
  \]
  Especially, for the down-set $U$ taken from~\ref{downsets-satur} we have
  \[
    d_L \cap (a \oplus b) \subseteq U.
  \]

  Following from~\ref{meets-in-satur}\thinspace, the saturated $\left(a \wedge
  b\right) \oplus \left(a \wedge b\right) \vee d_L \supseteq d_L$ contains $(a,
  b)$ (since it trivially contains $\left( a \wedge b, a \wedge b \right)$).
  Hence,
  \begin{align*}
    (a, b) &\in (a \oplus b) \; \cap \; ((a \wedge b) \oplus (a \wedge b) \vee d_L) \\
           &\subseteq ((a \wedge b) \oplus (a \wedge b)) \; \vee \; ((a \oplus b) \cap d_L) \\
           &\subseteq ((a \wedge b) \oplus (a \wedge b)) \; \vee \; U \\
           &= U.
  \end{align*}
  The first inclusion using~distributivity and the fact that $(a \wedge b)
  \oplus (a \wedge b) \subseteq a \oplus b$;
  the latter one by the observation from the beginning of this proof.
  The equality is seen from the~definition of~$U$:
  to be more specific, from $(a \wedge b) \oplus (a \wedge b) \subseteq U$.

  This leads to the long sought contradiction:
  $(a, b)\not\in \n$ since neither $a$ nor $b$ may equal $0$ (by $a \not\leq b,
  \, b \not\leq a$).
  Thus, the only options available are either $a \leq a \wedge b$ or $b \leq a
  \wedge b$.
  However, those enforce $a \leq b$ and $b \leq a$ respectively,
  a~contradiction. 
\end{proof}
