\chapter{Weak and strong Hausdorff}

In this chapter we will contemplate point-free analogies of the $T_2$
axiom---a~separation slightly intricate to imitate in frames.
Rather, two resembling conditions will be presented.

\section{The $T_2$ axiom}

\begin{framed}
  \begin{df}[$T_2$]
    A topological space $(X, \tau)$ is called \emph{Hausdorff\/} (or, as well,
    the~\emph{$T_2$-space\/}) whenever
    \begin{center} \it
      for any $x \ne y$ from~$X$ there are open $U\owns x$ and $V\owns y$
      disjoint from~each other.
    \end{center}
  \end{df}
\end{framed}

\begin{rem} \label{T2->T1}
  Trivially from the definitions every $T_2$-space is, additionally, $T_1$.
  Thus, the sets of the form $X\setminus \left\{x\right\}$ are among the
  elements of $\Omega(X)$ (viz \ref{T1Char}).
\end{rem}

\section{The Dowker-Strauss approach}

In 1972 Dowker and Strauss \cite{ds72} introduced the condition:

\begin{framed}
  \begin{df}[$S_2'$]
    A locale $L$ satisfies the $S_2'$ axiom if for any $a, b \ne 1$ we have
    \[
      a \vee b = 1 \qquad \Rightarrow \qquad \exists u\not\leq a, v\not\leq
      b: \quad u \wedge v = 0.
    \]
  \end{df}
\end{framed}

As we are about to discover, it extends the Hausdorff property.

\begin{prop} \label{Haus->S2'}
  For every Hausdorff space $X$ the corresponding locale $\Omega(X)$ is always
  $S_2'$.
\end{prop}

\begin{proof}
  Open sets $A, B \ne X$ with $A \cup B = X$ contain $x\in X\setminus A =
  B\setminus A$ and $y\in X\setminus B = A\setminus B$.
  Particularly, we observe $x \ne y$ and, on top of that, receive the disjoint
  $U$ and $V$ from the $T_2$-definition.
  Evidently, $U\not\subseteq A$ (because of $x$) and $V\not\subseteq B$
  (because of $y$).
\end{proof}

\begin{rem}
  The reverse implication requires, in addition, the subfitness.
  For the~proof seek Proposition 7 of~Dowker and Strauss
  \cite{ds72}.
\end{rem}

For our purposes, nevertheless, $S_2'$ will be replaced by a~somewhat weaker
description:

\begin{framed}
  \begin{df}[DS-Haus]
    A locale $L$ is considered to be \emph{weakly Hausdorff\/} (also
    \emph{DS-Hausdorff} as in \emph{Dowker-Straus-Hausdorff}) if
    \[
      a \vee b \not\in \left\{a, b\right\} \qquad \Rightarrow \qquad \exists
      u\not\leq a, v\not\leq b: \quad u \wedge v = 0.
    \]
  \end{df}
\end{framed}

\begin{rem} \label{DSHaus->Haus}
  The analogous observation as in~\ref{Haus->S2'} holds:
  the~premise
  \[
    A \cup B \not\in \left\{ A, B \right\}
  \]
  is translated to
  \[
    A \not\subseteq B \textrm{ and } B \not\subseteq A;
  \]
  the rest of the proof can easily be finished in the same way.
\end{rem}

\section{Isbell's adaptation}

For the sake of future clarity, we now reveal another characterization of
$T_2$:

\begin{prop}
  A topological space $X$ is Hausdorff space if and only if the~diagonal
  $\left\{(x, x) \st x\in X \right\}$ is closed.
\end{prop}

\begin{proof}
  Let $\Delta$ stand for the mentioned diagonal.

  $\Rightarrow:$ Any element $(x, y)\not\in \Delta$, that means $x \ne y$, is
  separable from $\Delta$:
  namely, by~the~open $p_1^{-1}[U] \cap p_2^{-1}[V]$ with non-intersecting $U$
  and $V$ from the definition of~$(T_2)$.
  Thence, $(x, y)\not\in \overline{\Delta}$ concludes to $\Delta =
  \overline{\Delta}$, which is closed.

  $\Leftarrow:$ Choose $x \ne y$.
  In such a case, we get a $W\owns (x, y)$ from $\Omega(X\times X)$ fulfilling
  $W \cap \Delta = \none$.
  Without loss of generality, let $W$ be from basis.
  That is, $W$ is a~finite intersection
  \[
    \bigcap_{i=1}^m p_1^{-1}[U_i] \cap \bigcap_{j=1}^n p_2^{-1}[V_i] =
    p_1^{-1}[\bigcap_{i=1}^m U_i] \cap p_2^{-1}[\bigcap_{j=1}^n V_i],
  \]
  further denoted only by $p_1^{-1}[U] \cap p_2^{-1}[V]$.

  Moreover, it is not possible for $W$ to be of the plain form $p_i^{-1}[U]$
  for $i = 1, 2$;
  since otherwise $p_i^{-1}[U] \cap \Delta \ne \none$, a contradiction.
  Following $W \cap \Delta = \none$, the open sets $U\owns x$ and $V\owns y$
  are disjoint from each other.
\end{proof}

Most likely, the just presented equivalence was the motivation for Isbell's
point-free approach to the $T_2$ axiom:
In~\cite{isbell72}, Hausdorff locales are exactly the ones where we have the
diagonal morphism (see Example~\ref{diag-in-Loc})
\[
  \Delta\colon L \to L \oplus L
\]
as a closed localic map.

Id est, the image~$\Delta[L]$ is a closed sublocale in~$L \oplus L$.

\begin{lem}
  $\bigwedge \Delta[L] = \Delta(0_L)$.
\end{lem}

\begin{proof}
  Again, recall Example~\ref{diag-in-Loc}\thinspace.

  $\subseteq$: The down-set $\Delta(0_L)$ is among elements of $\Delta[L]$; the
  infimum lies below all of~those.

  $\supseteq$: By the formula of the diagonal morphism,
  \[
    \Delta(0_L)
    = \left\{(x, y) \st x \wedge y \leq 0_L \right\}
    \subseteq \left\{(x, y) \st x \wedge y \leq a \right\}
    = \Delta(a)
  \]
  for every $a\in L$.
  From the property of the~infimum, $\Delta(0_L) \subseteq \bigwedge
  \Delta[L]$.
\end{proof}

Combining the above lemma with the~formula of~the sublocalic closure
\[
  \overline{S} = \left\uparrow \bigwedge \right. S,
\]
Isbell's condition reduces to
\[
  \Delta[L] = \left\uparrow \Delta(0_L),\right.
\]
which---after denoting the crucial element $\Delta(0_L)$ by $d_L$---produces
the final

\begin{framed}
  \begin{df}[I-Haus]
    A locale $L$ is considered \emph{strongly Hausdorff\/} (equally
    \emph{I-Hausdorff}: short for \emph{Isbell-Hausdorff}) iff
    \[
      \Delta[L] = \uparr d_L
    \]
  \end{df}
\end{framed}

\begin{prop} \label{IHaus->DSHaus}
  The (I-Haus) implies the (DS-Haus).
\end{prop}
(The strong version implies the weak one; hence the adjectives.)
\begin{proof}
  To be done.
\end{proof}
