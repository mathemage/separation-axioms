\chapter{Weak and strong Hausdorff}

In this chapter we will contemplate point-free analogies of the $T_2$
axiom---a~separation not trivial to imitate in frames.
Rather, two resembling conditions will be presented.

\section{The $T_2$ axiom}

\begin{framed}
  \begin{df}[$T_2$]
    A topological space $(X, \tau)$ is called \emph{Hausdorff\/} (or, as well,
    the~\emph{$T_2$-space\/}) whenever
    \begin{center} \it
      for any $x \ne y$ from~$X$ there are open $U\owns x$ and $V\owns y$
      disjoint from~each other.
    \end{center}
  \end{df}
\end{framed}

\section{The Dowker-Strauss adaptation}

\begin{framed}
  \begin{df}[$S_2'$]
    A locale $L$ satisfies the $S_2'$ axiom if for any $a, b \ne 1$ we have
    \[
      a \vee b = 1 \qquad \Rightarrow \qquad \exists u\not\leq a, v\not\leq
      b: \quad u \wedge v = 0.
    \]
  \end{df}
\end{framed}

This condition---introduced by Dowker and Strauss in 1972~\cite{ds72}---

\begin{framed}
  \begin{df}[DS-Haus]
    A locale $L$ is considered to be \emph{weakly Hausdorff\/} (also
    \emph{DS-Hausdorff} as in \emph{Dowker-Straus-Hausdorff}) if
    \[
      a \vee b \not\in \left\{a, b\right\} \qquad \Rightarrow \qquad \exists
      u\not\leq a, v\not\leq b: \quad u \wedge v = 0.
    \]
  \end{df}
\end{framed}

\section{Isbell's approach}
