\chapter{Weak and strong Hausdorff}

In this chapter we will contemplate point-free analogies of the $T_2$
axiom---a~separation slightly intricate to imitate in frames.
Rather, two resembling conditions will be presented.

\section{The $T_2$ axiom}

\begin{framed}
  \begin{df}[$T_2$]
    A topological space $(X, \tau)$ is called \emph{Hausdorff\/} (or, as well,
    the~\emph{$T_2$-space\/}) whenever
    \begin{center} \it
      for any $x \ne y$ from~$X$ there are open $U\owns x$ and $V\owns y$
      disjoint from~each other.
    \end{center}
  \end{df}
\end{framed}

\begin{rem} \label{T2->T1}
  Trivially from the definitions every $T_2$-space is, additionally, $T_1$.
  Thus, the sets of the form $X\setminus \left\{x\right\}$ are among the
  elements of $\Omega(X)$ (viz \ref{T1Char}).
\end{rem}

\section{The Dowker-Strauss approach}

In 1972 Dowker and Strauss \cite{ds72} introduced the condition:

\begin{framed}
  \begin{df}[$S_2'$]
    A locale $L$ satisfies the $S_2'$ axiom if for any $a, b \ne 1$ we have
    \[
      a \vee b = 1 \qquad \Rightarrow \qquad \exists u\not\leq a, v\not\leq
      b: \quad u \wedge v = 0.
    \]
  \end{df}
\end{framed}

As we are about to comprehend, it suitably ``mimics'' the Hausdorff property.

\begin{prop} \label{S2'->Haus}
  For $S_2'$-locale $\Omega(X)$ the corresponding space $X$ is always
  Hausdorff.
\end{prop}

\begin{proof}
  Recall~\ref{T2->T1}.
  Having open sets $a := X \setminus \left\{x\right\}$ and $b := X \setminus
  \left\{y\right\}$ with evident $a \cup b = X$, we acquire the $u$, $v$ from
  the $(S_2')$ definition.
  Since $u\not\subseteq X\setminus \left\{x\right\}$, the element $x$ has to
  belong to $u$.
  By symmetry, also $y \in v$.
\end{proof}

For our purposes, nevertheless, $S_2'$ will be replaced by a~somewhat weaker
description:

\begin{framed}
  \begin{df}[DS-Haus]
    A locale $L$ is considered to be \emph{weakly Hausdorff\/} (also
    \emph{DS-Hausdorff} as in \emph{Dowker-Straus-Hausdorff}) if
    \[
      a \vee b \not\in \left\{a, b\right\} \qquad \Rightarrow \qquad \exists
      u\not\leq a, v\not\leq b: \quad u \wedge v = 0.
    \]
  \end{df}
\end{framed}

\begin{rem}
  The analogous observation as in~Proposition~\ref{S2'->Haus} holds:
  because
  \[
    (X \setminus \left\{x\right\}) \cup (X \setminus \left\{y\right\}) = X
    \not\in \left\{X \setminus \left\{x\right\}, X \setminus \left\{y\right\}
    \right\},
  \]
  for the rest of the proof we can easily proceed in the same way.
\end{rem}

\section{Isbell's adaptation}
